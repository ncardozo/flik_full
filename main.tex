%%
% !TEX root = main.tex

\RequirePackage[final]{graphicx}
\documentclass[a4paper,fleqn]{cas-sc}

%\pagenumbering{arabic}
%\pagestyle{plain}

%\usepackage{lmodern}
\usepackage{exscale} 
\usepackage[T1]{fontenc}
\usepackage[utf8]{inputenc}

%---- PACKAGES
\usepackage{ifdraft}

\usepackage{amssymb}
\usepackage{amsmath}
\usepackage{mathrsfs}
\usepackage{hyperref}
\usepackage[plain]{fancyref}
\usepackage{subcaption}
\let\labelindent\relax
\usepackage[inline]{enumitem}
\usepackage{xcolor}
\usepackage{tikz}
\usepackage{booktabs}
\usepackage{threeparttable}
\usepackage{multirow}
\usepackage{xspace}
\usepackage[final]{listings}
\usepackage{acronym}
\usepackage{url}
\usepackage{balance}
\usepackage{algorithm}% http://ctan.org/pkg/algorithms
\usepackage{algpseudocode}% http://ctan.org/pkg/algorithmicx
\usepackage[numbers]{natbib}

\marginparwidth=0.9cm


\usepackage{etoolbox}
\makeatletter
\patchcmd{\@makecaption}
  {\scshape}
  {}
  {}
  {}
\patchcmd{\@makecaption}
  {\\}
  {.\ }
  {}
  {}


\let\refsection\relax
%\usepackage
%  [backend=biber,
%%   natbib=true,
%   style=elsarticle-num-names,
%   maxbibnames=10,
%   minbibnames=5,
%   maxcitenames=1,
%%   giveninits=true,
%   hyperref=true,
%   url=true,
%   defernumbers]{biblatex}
%
%
%%----[ Biber ]----
%\addbibresource[datatype=bibtex]{local.bib}
%\addbibresource[datatype=bibtex]{bib/strings.bib}
%\addbibresource[datatype=bibtex]{bib/general.bib}
%\addbibresource[datatype=bibtex]{bib/compsci.bib}
%\addbibresource[datatype=bibtex]{bib/learning.bib}
%
%\AtEveryBibitem
%  {\clearlist{address}
%   \clearfield{date}
%   \clearfield{doi}
%   \clearfield{eprint}
%   \clearfield{isbn}
%   \clearfield{issn}
%   \clearfield{month}
%   \clearfield{note}
%   \clearfield{pages}
%   \clearlist{location}
%   \clearfield{series}
%   \clearfield{url}
%   \clearname{editor}
%   \ifentrytype{inproceedings}
%     {\clearfield{day}
%      \clearfield{month}
%      \clearfield{volume}}{}}
%
%\DeclareFieldFormat*{title}{\textsl{#1}\isdot}
%\DeclareFieldFormat*{journaltitle}{#1}
%\DeclareFieldFormat*{booktitle}{#1}
%
%\renewbibmacro{in:}{} % supress 'In: ' form
%
%\DeclareSourcemap
% {\maps[datatype=bibtex,overwrite]
%   {% Tag entries (through keywords)
%    \map
%      {\step[fieldsource=booktitle,
%       match=\regexp{[Pp]roceedings}, replace={Proc.}]}
%    \map
%      {\step[fieldsource=booktitle,
%       match=\regexp{[Ii]nternational}, replace={Intl.}]}
%   \map
%      {\step[fieldsource=booktitle,
%       match=\regexp{[Cc]onference}, replace={Conf.}]}
%    \map
%      {\step[fieldsource=booktitle,
%       match=\regexp{[Ss]ymposium}, replace={Symp.}]}
%    \map
%      {\step[fieldsource=journal,
%       match=\regexp{[Jj]ournal}, replace={Jour.}]}
%    \map
%      {\step[fieldsource=journal,
%       match=\regexp{[Tt]ransactions}, replace={Trans.}]}
%	\map
%      {\step[fieldsource=journal,
%       match=\regexp{[Aa]nnals}, replace={Ann.}]}
%	\map
%      {\step[fieldsource=journal,
%       match=\regexp{[Ss]oftware}, replace={Soft.}]}
%	\map
%      {\step[fieldsource=journal,
%       match=\regexp{[Ee]ngineering}, replace={Eng.}]}
%    \map
%      {\step[fieldsource=booktitle,
%       match=\regexp{[Pp]roceedings\s+of\s+the.+[Ee]uropean\s+[Cc]onference\s+in}, replace={European Conf. in}]}
%    \map
%      {\step[fieldsource=booktitle,
%       match=\regexp{In\s+[Pp]roceedings\s+of\s+the\s+[Ss]ymposium\s+on}, replace={Symp. on}]}
%     \map
%      {\step[fieldsource=publisher,
%       match=\regexp{[Aa]ssociation\s+for\s+[Cc]omputing\s+[Mm]achinery\s}, replace={ACM}]}
%     \map
%      {\step[fieldsource=booktitle,
%       match=\regexp{[Pp]roceedings\s+of\s+the\s+[Ii]nternational\s+[Cc]onference\s+on}, replace={Intl. Conf. on}]}
%    \map
%      {\step[fieldsource=booktitle,
%       match=\regexp{[Pp]roceedings\s+of\s+the\s+[Ii]nternational\s+[Ww]orkshop\s+on}, replace={Intl. Workshop on}]}}}



%color
\definecolor{OliveGreen}{rgb}{0,0.6,0.3}

%References
%% Listings
\def\fref{\Fref} % treat all \frefs as \Frefs
\renewcommand{\lstlistingname}{Snippet}
\newcommand*{\fancyreflstlabelprefix}{lst}
\newcommand*{\Freflstname}{\lstlistingname}
\newcommand*{\freflstname}{\MakeLowercase{\lstlistingname}}
\Frefformat{vario}{\fancyreflstlabelprefix}%
  {\Freflstname\fancyrefdefaultspacing#1#3}
\frefformat{vario}{\fancyreflstlabelprefix}%
  {\freflstname\fancyrefdefaultspacing#1#3}
\Frefformat{plain}{\fancyreflstlabelprefix}%
  {\Freflstname\fancyrefdefaultspacing#1}
\frefformat{plain}{\fancyreflstlabelprefix}%
  {\freflstname\fancyrefdefaultspacing#1}

\renewcommand{\tablename}{Table}  
\renewcommand{\figurename}{Figure}  
  
% ln delimiter
\newcommand*{\fancyreflnlabelprefix}{ln}
\newcommand*{\Freflnname}{Line}
\newcommand*{\freflnname}{\MakeLowercase{\Freflnname}}
\Frefformat{vario}{\fancyreflnlabelprefix}%
  {\Freflnname\fancyrefdefaultspacing#1#3}
\frefformat{vario}{\fancyreflnlabelprefix}%
  {\freflnname\fancyrefdefaultspacing#1#3}
\Frefformat{plain}{\fancyreflnlabelprefix}%
  {\Freflnname\fancyrefdefaultspacing#1}
\frefformat{plain}{\fancyreflnlabelprefix}%
  {\freflnname\fancyrefdefaultspacing#1}    

%def delimiter
\newcommand*{\fancyrefdeflabelprefix}{def}
\newcommand*{\Frefdefname}{Definition}
\newcommand*{\frefdefname}{\MakeLowercase{\Frefdefname}}
\Frefformat{vario}{\fancyrefdeflabelprefix}%
  {\Frefdefname\fancyrefdefaultspacing#1#3}
\frefformat{vario}{\fancyrefdeflabelprefix}%
  {\frefdefname\fancyrefdefaultspacing#1#3}
\Frefformat{plain}{\fancyrefdeflabelprefix}%
  {\Frefdefname\fancyrefdefaultspacing#1}
\frefformat{plain}{\fancyrefdeflabelprefix}%
  {\frefdefname\fancyrefdefaultspacing#1}    
  
%theorem delimiter
\newcommand*{\fancyreftheolabelprefix}{theo}
\newcommand*{\Freftheoname}{Theorem}
\newcommand*{\freftheoname}{\MakeLowercase{\Freftheoname}}
\Frefformat{vario}{\fancyreftheolabelprefix}%
  {\Freftheoname\fancyrefdefaultspacing#1#3}
\frefformat{vario}{\fancyreftheolabelprefix}%
  {\freftheoname\fancyrefdefaultspacing#1#3}
\Frefformat{plain}{\fancyreftheolabelprefix}%
  {\Freftheoname\fancyrefdefaultspacing#1}
\frefformat{plain}{\fancyreftheolabelprefix}%
  {\freftheoname\fancyrefdefaultspacing#1}    

%Proposition delimiter
\newcommand*{\fancyrefproplabelprefix}{prop}
\newcommand*{\Frefpropname}{Proposition}
\newcommand*{\frefpropname}{\MakeLowercase{\Frefpropname}}
\Frefformat{vario}{\fancyrefproplabelprefix}%
  {\Frefpropname\fancyrefdefaultspacing#1#3}
\frefformat{vario}{\fancyrefproplabelprefix}%
  {\frefpropname\fancyrefdefaultspacing#1#3}
\Frefformat{plain}{\fancyrefproplabelprefix}%
  {\Frefpropname\fancyrefdefaultspacing#1}
\frefformat{plain}{\fancyrefproplabelprefix}%
  {\frefpropname\fancyrefdefaultspacing#1}    
  
  
%Code environment definition
\lstset{%
  basicstyle=\footnotesize\ttfamily,
  aboveskip=0\baselineskip,
  belowskip=0\baselineskip,
  commentstyle=\scriptsize\itshape,
  keywordstyle=\ttfamily\bfseries,
  breaklines,
  numberblanklines=false,
  numberstyle=\tiny\color{gray}, 
  commentstyle=\color{gray},
  numbersep=0pt,
  escapechar=`,  
  numberbychapter=false}
  
\lstdefinestyle{floating}
 {frame=lines,
  float=hptb,
  captionpos=b,
  abovecaptionskip=-0pt}

% python listings
\lstdefinestyle{py}
 {language=Python,
  showstringspaces=false,
  tabsize=2,
  style=floating,
  belowskip=-0\baselineskip,
  aboveskip=-0\baselineskip,
  keywords=[2]{def, for, while, if, else, return, self},
%  deletekeywords={len, max, range},
  keywordstyle=[2]\ttfamily\bfseries,
  keywords=[1]{breakpoint,n,p,pp,s,c,step_back,back_to,variables,sticky,args,setvar,unt,b,w,u,d,help,q},
  keywordstyle=[1]\ttfamily\bfseries\color{blue},
  deletekeywords={len,max,range},
}

%context traits environment    
 \lstnewenvironment{python}[1][]
 {\lstset{style=py,#1}}{}  

 % Context Traits in line source-code
\newcommand{\spy}[1]{\lstinline[style=py]{#1}}


%%Boxes
\usepackage[many]{tcolorbox} 
\usepackage{setspace} 

\setlength\parindent{0pt}   % killing indentation for all the text
\setstretch{1}            % setting line spacing to 1.3
%\setlength\columnsep{0.1in} % setting length of column separator
\pagestyle{empty}           % setting pagestyle to be empty


\definecolor{main}{HTML}{5989cf}    % setting main color to be used
\definecolor{sub}{HTML}{cde4ff}     % setting sub color to be used

\tcbset{
    sharp corners,
    colback = white,
    before skip = 0.1cm,    % add extra space before the box
    after skip = 0.2cm      % add extra space after the box
} 

\newtcolorbox{highlight}{
    colback = sub, 
    colframe = main, 
    boxrule = 0pt, 
    leftrule = 3pt % left rule weight}
}

%----[ Figures ]---
%\addtolength{\intextsep}{-1mm}

%%captions
%\addtolength{\abovecaptionskip}{-0mm}
%\addtolength{\belowcaptionskip}{-0mm}
%\captionsetup[figure]{aboveskip=-0ex,belowskip-0ex}
%\captionsetup[table]{aboveskip=-0ex,belowskip-0ex}
%\captionsetup[lstlisting]{aboveskip=-1ex,belowskip=-1ex}

%DPN
%\usepackage[version=0.96]{pgf}
%\usetikzlibrary{arrows, shapes, backgrounds}
%\usetikzlibrary{decorations.pathreplacing}
%\usetikzlibrary{shapes.misc}
%\usetikzlibrary{petri}

%% Petri nets
\tikzstyle{place}=[circle,thick,draw=black!75,minimum size=5mm]
\tikzstyle{iplace}=[circle,dashed,thick,draw=black!75,minimum size=5mm]
\tikzstyle{itransition}=[rectangle,draw,thick,fill=black,minimum size=1mm]
\tikzstyle{etransition}=[rectangle,draw,thick,minimum size=1mm]
\tikzstyle{ctransition}=[rectangle,draw,color=black!45,thick,fill=black!45,minimum size=1mm]

\tikzstyle{dpn}=
 [node distance=1.3cm, >=stealth', bend angle=45, auto,
  font=\fontsize{8}{8}\selectfont]

%----[ Commands ]---
%Latins
\newcommand{\eg}{\emph{e.g.,}\xspace}
\newcommand{\ie}{\emph{i.e.,}\xspace}
\newcommand{\cf}{\emph{cf.}\xspace}

\newcommand{\ctx}[1]{\texttt{\textsc{#1}}}

%petri
\newcommand{\enabled}[2]{$#1[#2\rangle$}
%names
\newcommand{\flik}{\textsc{Flik}\xspace}

%comments
% xcolor
\definecolor{author}{rgb}{.5, .5, .5}
\definecolor{comment}{rgb}{.1, .0, .9}
\definecolor{note}{rgb}{.9, .4, .0}
\definecolor{idea}{rgb}{.1, .7, .0}
\definecolor{missing}{rgb}{.9, .1, .0}


\newcommand{\authorcomment}[3][comment]
  {\ifdraft{\noindent
      \fbox{\footnotesize\textcolor{author}{\textsc{#2}}}
      \textcolor{#1}{\textsl{#3}}}{}}

% $Id: acronyms.tex  $
% !TEX root = main.tex

\acrodef{RL}{Reinforcement Learning}
\acrodef{AI}{Artificial Intelligence}
\acrodef{ML}{Machine Learning}
\acrodef{GDB}{GNU Debugger}
\acrodef{IDE}{Integrated Development Environment}
\acrodef{API}{Application Programming Interface}
\acrodef{DDD}{Data-Display Debugger}
\acrodef{PDB}{Python Debugger}
\acrodef{GUI}{Graphical User Interface}
\acrodef{RevPDB}{Reverse Python Debugger}
\acrodef{UDB}{Undo Debugger}
\acrodef{Vizarel}{A Tool for Interactive Visualization of Reinforcement Learning}
\acrodef{MDP}{Markov Decision Process}
\acrodef{TD}{Temporal-Difference}

%---[ Acronyms Plurals Special Forms] ---


%\newcommand{\acResetNonTrivial}
  

%\acResetNonTrivial
{\acresetall
   \acused{CPU}
   \acused{VAR}
   \acused{AST}
   \acused{API}
   \acused{LAN}
   \acused{SMT}
   \acused{GUI}}


\endinput

%Space squeezing
\let\orig@figure\figure
\renewcommand*{\figure}[1][]{\orig@figure[#1]\vspace{-1ex}} % before
\let\orig@endfigure\endfigure
\renewcommand*{\endfigure}{\vspace{-1.5ex}\orig@endfigure} % after

\let\orig@endlstlisting\endlstlisting
\renewcommand*{\endlstlisting}{\vspace{-3ex}\orig@endlstlisting} % after

% Squeeze captions
\usepackage{caption}
\captionsetup[figure]{aboveskip=0.0em,belowskip=-0em}
\captionsetup[table]{aboveskip=0em,belowskip=-0em}



\makeatother
% $Id: acronyms.tex  $
% !TEX root = main.tex

\acrodef{RL}{Reinforcement Learning}
\acrodef{AI}{Artificial Intelligence}
\acrodef{ML}{Machine Learning}
\acrodef{GDB}{GNU Debugger}
\acrodef{IDE}{Integrated Development Environment}
\acrodef{API}{Application Programming Interface}
\acrodef{DDD}{Data-Display Debugger}
\acrodef{PDB}{Python Debugger}
\acrodef{GUI}{Graphical User Interface}
\acrodef{RevPDB}{Reverse Python Debugger}
\acrodef{UDB}{Undo Debugger}
\acrodef{Vizarel}{A Tool for Interactive Visualization of Reinforcement Learning}
\acrodef{MDP}{Markov Decision Process}
\acrodef{TD}{Temporal-Difference}

%---[ Acronyms Plurals Special Forms] ---


%\newcommand{\acResetNonTrivial}
  

%\acResetNonTrivial
{\acresetall
   \acused{CPU}
   \acused{VAR}
   \acused{AST}
   \acused{API}
   \acused{LAN}
   \acused{SMT}
   \acused{GUI}}


\endinput
%% Reviewing.
\usepackage{newfile}
\usepackage{pgf, tikz}

%Side margins
\usepackage[fulladjust]{marginnote}

\newcounter{marginEnum}

\edef\marginnotetextwidth{\the\textwidth}

\newcommand{\sidemargin}[2]{
      \refstepcounter{marginEnum}
      \begingroup%
      \label{#1}
      \marginnote{ 
        	\includegraphics[width=0.5\marginparwidth]{figures/fix}
        	\textbf{\themarginEnum.} 
        	#2}[0cm]
      \endgroup%
\xspace
}

\newcommand{\lsidemargin}[2]{
      \refstepcounter{marginEnum}
      \begingroup%
      \label{#1}
	\hspace{-3.5cm}
        \marginnote{ 
        	\includegraphics[width=0.5\marginparwidth]{figures/fix}
        	\textbf{\themarginEnum.} 
        	#2}[0cm]
      \endgroup%
\xspace
}


%References
%% Remarks
\newcommand*{\fancyrefremlabelprefix}{rem}
\newcommand*{\Frefremname}{Remark}
\newcommand*{\frefremname}{\MakeLowercase{\Frefremname}}
\Frefformat{vario}{\fancyrefremlabelprefix}%
  {\Frefremname\fancyrefdefaultspacing#1#3}
\frefformat{vario}{\fancyrefremlabelprefix}%
  {\frefremname\fancyrefdefaultspacing#1#3}
\Frefformat{plain}{\fancyrefremlabelprefix}%
  {\Frefremname\fancyrefdefaultspacing#1}
\frefformat{plain}{\fancyrefremlabelprefix}%
  {\frefremname\fancyrefdefaultspacing#1}   

% par delimiter
\newcommand*{\fancyrefparlabelprefix}{par}
\newcommand*{\Frefparname}{Margin note}
\newcommand*{\frefparnname}{\MakeLowercase{\Frefparname}}
\Frefformat{vario}{\fancyrefparlabelprefix}%
  {\Frefparname\fancyrefdefaultspacing#1#3}
\frefformat{vario}{\fancyrefparlabelprefix}%
  {\frefparname\fancyrefdefaultspacing#1#3}
\Frefformat{plain}{\fancyrefparlabelprefix}%
  {\Frefparname\fancyrefdefaultspacing#1}
\frefformat{plain}{\fancyrefparlabelprefix}%
  {\frefparname\fancyrefdefaultspacing#1} 

\newcounter{remNumber}
\newenvironment{remark}[1][Remark]{
\addtocounter{remarks}{1}%
\refstepcounter{remNumber} \begin{trivlist}
%\newenvironment{remark}[1][Remark]{\addtocounter{remarks}{1}\begin{trivlist}
\item[\hskip \labelsep {\bfseries\itshape #1 \theremNumber:}]}{\end{trivlist}}

\colorlet{quote}{black!60}
\colorlet{remark}{black!80}

% \colorlet{unsolved}{red!90!black}
\definecolor{solved}{hsb}    { .35, 1, .7}
\definecolor{future}{hsb}    { .50, 1, .7}
\definecolor{incorrect}{hsb} { .65, 1, .7}
\definecolor{disagree}{hsb}  { .80, 1, .9}
\definecolor{unsolved}{hsb}  { .95, 1, .7}
\definecolor{unspecific}{hsb}{ .10, 1, .8}


\newcommand{\status}[2]{\textsf{[\,\textcolor{#1}{#2}\,]}}
\newcommand{\statussolved}{\status{solved}{OK}}
\newcommand{\statusfuture}{\status{future}{Future work}}
\newcommand{\statusincorrect}{\status{incorrect}{Clarification}}
\newcommand{\statusdisagree}{\status{disagree}{Reconsider}}
\newcommand{\statusunsolved}{\status{unsolved}{Blocked}}
\newcommand{\statusunspecific}{\status{unspecific}{Feedback}}


%\makeatletter
%\newcommand*{\@doendeq}
% {\everypar{{\setbox\z@\lastbox}\everypar{}}}
%\newenvironment{rev_remark}[1]
%  {\begingroup%
%   \color{rev_remark}%
%   \def\FrameCommand{\vrule width 1pt \hspace{10pt}}%
%   \MakeFramed {\advance\hsize-\width \FrameRestore}%
%   \slshape\noindent%
%   #1}
%  {\endMakeFramed%
%   \endgroup\ignorespacesafterend\aftergroup\@doendeq}
%\makeatother

\newenvironment{incorrect}{\addtocounter{incorrect}{1}\statusincorrect}{}
\newenvironment{unsolved}{\addtocounter{unsolved}{1}\statusunsolved}{}
\newenvironment{disagree}{\addtocounter{disagree}{1}\statusdisagree}{}
\newenvironment{unspecific}{\addtocounter{unspecific}{1}\statusunspecific}{}
\newenvironment{future}{\addtocounter{future}{1}\statusfuture}{}
\newenvironment{solved}{\addtocounter{solved}{1}\statussolved}{}

\def\flags{remarks,solved,unsolved,incorrect,disagree,unspecific,future}

\foreach \flag in \flags
  {\expandafter\newcounter{\flag}
   \expandafter\newcounter{total\flag}}

\InputIfFileExists{\jobname.cnt}{}{}

\newcounter{totalresponses}
\defcounter{totalresponses}%
  {\value{totalincorrect}  +
   \value{totalunsolved} +
   \value{totaldisagree} +
   \value{totalunspecific} +
   \value{totalfuture} +
   \value{totalsolved} }

\newcommand{\revtotal}[1] {
	\makebox[3ex][r]{\arabic{total#1}}
	\if\value{#1}1 
     	remark is flagged as \csname status#1 \endcsname
	\else
		remarks are flagged as \csname status#1 \endcsname
	\fi
} 


\newoutputstream{counters}
\openoutputfile{\jobname.cnt}{counters}

\AtEndDocument
  {\foreach \flag in \flags
     {\addtostream{counters}%
        {\noexpand\setcounter{total\flag}{\arabic{\flag}}}}
   \closeoutputstream{counters}}
   
\usepackage[]{lineno}
\linenumbers

\begin{document}
 \let\WriteBookmarks\relax
\def\floatpagepagefraction{1}
\def\textpagefraction{.001}


% Short title
\shorttitle{Flik}    

% Short author
\shortauthors{L. Rodriguez et al.}  

% Main title of the paper
\title[mode = title]{Flik: A Back-in-time Debugger for Reinforcement Learning Programs}  

% Title footnote mark
% eg: \tnotemark[1]
%\tnotemark[1] 

% Title footnote 1.
% eg: \tnotetext[1]{Title footnote text}
%\tnotetext[1]{} 

% First author
%
% Options: Use if required
% eg: \author[1,3]{Author Name}[type=editor,
%       style=chinese,
%       auid=000,
%       bioid=1,
%       prefix=Sir,
%       orcid=0000-0000-0000-0000,
%       facebook=<facebook id>,
%       twitter=<twitter id>,
%       linkedin=<linkedin id>,
%       gplus=<gplus id>]

\author[1]{Laura {Rodriguez}}[orcid=0009-0000-9715-0475]
\ead{la.rodriguez@uniandes.edu.co}

\author[1]{Mario {Linares-V\'asquez}}[orcid=0000-0003-0161-2888]
\ead{m.linarev@uniandes.edu.co}

\author[2]{{Ivana} {Dusparic}}[orcid=0000-0003-0621-5400]
\ead{ivana.dusparic@tcd.ie}

\author[1]{{Nicol\'as} {Cardozo}}[orcid=0000-0002-1094-9952]
\ead{n.cardozo@uniandes.edu.co}
\cormark[1]

% Address/affiliation
\affiliation[1]{organization={Department of Systems and Computing Engineering, Universidad de los Andes}, 
	city={Bogota}, 
	country={Colombia}}

\affiliation[2]{organization={School of Computer Science and Statistics, Trinity College Dublin}, 		city={Dublin}, 
	country={Ireland}}


% Corresponding author text
\cortext[1]{Corresponding author}



% For a title note without a number/mark
%\nonumnote{}

% Here goes the abstract
\begin{abstract}
%Why
\textbf{Purpose:} \ac{RL}-based programs' have a high-level of complexity due to the size and 
conditions of the environment, and agents' interaction with it. Developers face challenges  
understanding the behavior of \ac{RL} agents and identifying bugs, and their cause, within the 
program execution. This problem is particularly challenging for bugs in which the program runs 
without errors, but the execution does not present an appropriate agent behavior.
%What
\textbf{Method:} To address this problem, we propose \flik, a back-in-time debugger for \ac{RL} 
programs. \flik is a console-based debugger that allows developers to interact with 
the program during execution, inspect the internal state of the agent, and modify 
its behavior in real-time. Changes to the agent generate alternative execution paths. Based on such 
execution path, developers can more easily identify bugs in their programs.
\textbf{Results:} We evaluate the usability and usefulness of \flik in identifying and helping to solve 
bugs by means of a user evaluation in which 27 developers interacted and solve bugs in three 
different \ac{RL}  programs. Our results show a positive response to \flik, highlighting that the 
debugger is useful in identifying the kinds of bugs that commonly arise in \ac{RL} programs. 
\textbf{Conclusion:} 
With the advent of \ac{RL} programs, software engineering and analysis tools are need to help 
assure their correct execution. This work presents a back-in-time debugger with functionality to 
update agents' state during their execution to be able to observe behavioral changes immediately, 
without having to re-execute the entire agent. The debugger proved useful through a user study 
confirming the importance of such tools.
\end{abstract}

% Use if graphical abstract is present
%\begin{graphicalabstract}
%\includegraphics{}
%\end{graphicalabstract}

% Research highlights
\begin{highlights}
\item Presentation of a first debugger for Reinforcement learning programs, based on back-in-time debugging capabilities.
\item Full implementation of a text/console-based debugger with the capability to: step through the code; observe variables' state; modify variables' state during execution to branch different execution paths; and step back in the execution restoring the state and execution context (stack instructions), again enabling the possibility to branch execution with different contexts.
\item Empirical study with 27 participants, developed to evaluate the usefulness and usability of the debugger in identifying bugs in three different Q-learning programs.
\end{highlights}


% Keywords
% Each keyword is seperated by \sep
\begin{keywords}
Debuggers \sep
Back-in-time \sep 
Reinforcement Learning \sep 
Empirical Study
\end{keywords}


\maketitle              % typeset the header of the contribution

% $Id: problem.tex 
% !TEX root = ../main.tex

\section{Introduction}
\label{sec:intro}

\sidemargin{par:intro}{{\fref{rem:intro}}}
The behavior of \ac{RL}~\cite{sutton18} agents emerges through the interaction of the agents with 
their deployment environment. In such interaction, agents execute actions sequences moving 
between environment states, collecting information about how pertinent are the actions to the 
agents objective, in the form of a scalar reward function. From interactions, agents learn a 
state-action function defining the optimal agent policy; the actions to take at each state in order to 
reach the agent's goal.

%why
\ac{RL} agents are commonly built based on the Q-learning algorithm~\cite{beakcheol19} (or 
\ac{DQN}~\cite{fan20} for continuous environments), driving the agent-environment interaction, 
and ultimately solving the \ac{MDP} problem describing the learning process for agents as a 
black-box. Common to software development processes, the development of \ac{RL} programs, is 
prone to errors or unexpected behavior in the application logic like incorrect state values, or 
erroneous logic implementations. Moreover, within \ac{RL} programs also appear errors in the 
definition of the learning hyperparameters, and incomplete or wrong definition of the reward 
function. 

Detecting erroneous behavior in \ac{RL} programs is a complicated task due to two main reasons. 
First, like \ac{ML} programs, \ac{RL} algorithms are usually used as a black box. This means that 
developers do not have access to the internals of the agent, the environment, or the learning 
algorithms. This poses a problem in many fields, as it often becomes difficult to identify why an 
fails to converge, or why it does not reach an optimal or expected solution as anticipated by the 
developer. In the case of \ac{RL}, this issue is particularly pronounced because it is challenging to 
determine whether the agent is learning correctly or if it is learning an inappropriate policy.
Second, \ac{RL} programs are notoriously cumbersome and difficult to debug due to the extensive 
training time required by the agent, demanding substantial computational resources. This 
complexity makes it incredibly challenging to locate errors, even when using standard 
debugging tools. 

Traditional debugging tools are often inadequate for tracing issues in \ac{RL} due to the intricate 
and prolonged nature of the training processes. In conventional software systems, the execution flow 
is typically linear, allowing developers to trace and debug step-by-step with relative ease. However, 
\ac{RL} programs involve a continuous loop of learning and adaptation, where an agent 
interacts with an environment stochastically, receives feedback from the interaction (in the form of 
scalar rewards), and adjusts its actions accordingly. The cyclical and stochastic nature of \ac{RL} 
programs, makes the process of identifying errors much more complex and less transparent.

As a result, understanding and debugging \ac{RL} programs requires tracking individual 
decisions, and understanding the long-term effects and patterns that emerge over many iterations.
Moreover, the need for substantial computational resources and extended training 
periods adds to the challenge. The iterative process can take hours, days, or 
even weeks, making it impractical to simply restart the training from scratch 
each time an error is detected.

The aforementioned problems and characteristics of program execution in \ac{RL} posit a need 
for advanced tools and methods for analyzing \ac{RL} agents' behavior, monitoring its learning 
progress, and pinpointing issues without having to re-run lengthy training sessions.
Addressing these challenges is crucial for advancing the state of \ac{RL} and increasing the quality 
of \ac{RL} agents, making \ac{RL}-based applications more reliable and interpretable. This is 
currently a gap in the literature, which we address as the focus of this work.

To address the aforementioned problems, this work develops a back-in-time debugger for the 
identification of errors or undesired behavior of \ac{RL} agents. The proposed debugger, \flik, is a 
console-based debugger with features to: 
\begin{enumerate*}[label=(\arabic*)]
\item record the history of the execution (both values and execution context) as it steps through the 
code,
\item go back to a previous execution point, reverting the state and execution context from the 
recorded history, and 
\item observe and change the state of the program to generate new execution traces from a particular 
point in time.
\end{enumerate*}
Given these features, developers can analyze \ac{RL} programs while they execute, enabling the 
experimentation of different state values or instruction sequences, without having to incur in long 
train-test processes for each possible configuration. With \flik, developers can try a particular agent 
configuration, go back in the execution, try another configuration and observe if the observed 
behavior is more appropriate or as expected. Such capabilities are useful in pinpointing the root 
cause of undesired behavior. 

To validate the usefulness in detecting bugs, and the usability of the tool itself, we conduct a user 
study with a focused group of 27 participants, all having previous Python experience and having 
taken a graduate course in \ac{RL}. The evaluation of \flik consists of three tasks representing 
different types of errors that may arise when developing \ac{RL} programs. Each task concerns a 
different \ac{RL} environment, increasing and complexity and less familiar to participants as the 
tasks progress, as to reduce possible evaluation bias or learning curve bias. Each of the tasks use 
a different environment, namely GridWorld, Rooms, and Driving Assistant. Our results show that 
the participants posit \flik as a useful tool to debug and understand the behavior of \ac{RL} agents 
more deeply. The participants note that, while usable, there are possible improvements in 
\flik's interface to improve usability.

The reminder of the paper is structured as follows. \fref{sec:background} presents a description of 
\ac{RL} and the Q-learning algorithm. \fref{sec:state_of_the_art} presents the state of the art with 
respect to general debugger definitions (\fref{sec:deb}), existing back-in-time debuggers and their 
functioning, in different programming languages (\fref{sec:other}), the presentation of debuggers 
and debugging capabilities for Python (\fref{sec:py}), and debuggers or debugger-like tools for 
\ac{RL} programs (\fref{sec:ai}). \fref{sec:solution} presents the rationale and details in \flik's design, 
as well as a running example of its back-in-time debugging capabilities. \fref{sec:evaluation} presents 
the details of the user study, and the three programs used for the evaluation (\ie GridWorld, Rooms, 
and Driving Assistant). \fref{sec:results} presents the results of the evaluation and the feedback from 
the participants, divided in the three main parts of the evaluation survey: general results 
(\fref{sec:general-knowledge}), tasks results (\fref{sec:tasks-results}) and usability results 
(\fref{sec:usability}). It also adds a final part for the discussion of the results (\fref{sec:discussion}). 
Finally, \fref{sec:conclusion} presents the conclusion and avenues of future work.


\endinput


% $Id: problem.tex 
% !TEX root = ../main.tex

\section{\acl{RL} Basics}
\label{sec:background}

\sidemargin{par:background}{{\fref{rem:intro}}}
\ac{RL}~\cite{sutton18} encompasses different artificial intelligence techniques and algorithms in 
which an agent learns the behavior to reach a goal by specializing its interaction with the 
environment in which it is deployed.
\ac{RL} agents learn the state-action function for a given policy through interaction with the 
environment to discover an optimal solution to reach a goal, through the optimal policy $\pi^*$ (a set 
of actions to execute at each system state) gathered from the interaction of the agent with the 
environment. One of the most common algorithms in \ac{RL} is Q-learning~\cite{beakcheol19}, which 
uses an off-policy control that separates the deferral policy from the learning policy and updates the 
action selection using the Bellman optimal equation, and the $\epsilon$-greed policy.  In the definition 
of \ac{RL} algorithms, the environment is composed of a set of states $S$, a set of actions $A$ the agent can execute. By performing an action $a\in A$ at state $s \in S$, the agent transitions from $s$
to a new state $s' \in S$. Additionally, associated to the execution of the action $a$ at state $s$, the 
agent receives a reward $r$. The goal of the agent is to maximize its total reward. \fref{lst:qlearning} 
presents the base (pseudo) algorithm for Q-learning~\cite{sutton18}.

\begin{algorithm}
\caption{Q-Learning Algorithm}\label{lst:qlearning}
\begin{algorithmic}
\Require An environment with states $S$, actions $A$, and reward function $R(s, a)$
\Require A learning rate $\alpha \in [0, 1]$
\Require A discount factor $\gamma \in [0, 1]$
\Require Exploration strategy (\eg $\epsilon$-greedy)
\State Initialize Q-table $Q(s, a) \gets 0$ for all $s \in S$, $a \in A$
\For{each episode}
    \State Initialize the starting state $s_0$
    \For{each time step in the episode}
        \If{exploration step}
            \State Choose an action $a$ using the exploration strategy (e.g., $\epsilon$-greedy)
        \Else
            \State Choose an action $a$ based on the current Q-values: $a = \arg\max_a Q(s, a)$
        \EndIf
        \State Take action $a$, observe reward $R(s,a)$ and next state $s'$
        \State Update Q-value: 
        \[
  \qquad  Q(s, a) \gets Q(s, a) + \alpha \left( R(s,a) + \gamma \max_{a'} Q(s', a')\right)
        \]
        \State Set $s \gets s'$
    \EndFor
\EndFor
\end{algorithmic}
\end{algorithm}

\ac{RL} has gained wide popularity in the last couple of years, finding applications in a wide range of 
domains~\cite{beakcheol19}, particularly in control of industrial process~\cite{kiumarsi18} (improving 
the performance of the on-line learning control system or optimizing temperature control and power 
consumption), computer networks~\cite{alrawi13} (improving adaptability of Wireless Sensor Networks 
to changing situations and eliminating the need for system redesign), and robotics~\cite{zhang15} (by 
providing frameworks and toolkits for designing sophisticated behavioral aspects). 




\endinput


% $Id: state_of_the_art.tex 
% !TEX root = ../main.tex

\section{State of the Art}
\label{sec:state_of_the_art}

This section presents the state of the art from three perspectives:, the main debuggers in other 
programming languages, existing Python debuggers, debuggers for \ac{AI} programs in general, 
including debuggers for \ac{RL} programs specifically.


%%
\subsection{Debuggers for other programming languages}
\label{sec:other}

We first turn our attention to the state of the art in debugging tools available for different 
programming languages. There are many debugging tools available for different programming 
languages, each with its own set of features and capabilities, most of them integrated with IDEs as
built-in debugging tools that allow developers to inspect programs' behavior during their execution. 
However, there are also standalone debugging tools that can be used with different 
programming languages, and that offer more advanced features and capabilities than the 
built-in debugging tools of IDEs. 

Some of the most popular debugging tools include \ac{GDB}~\cite{stallman11}, 
which stands for GNU Project Debugger and is a powerful 
debugging tool for C and C++ (although it also supports other programming languages,
like Ada, Go, or Rust). GDB helps developers to inspect the internals of C programs while 
they are executing. \ac{GDB} works on binary executable files produced during compilation. During 
execution, \ac{GDB} allows developers to observe exactly what happens when a program 
crashes. \ac{GDB} consists of three major subsystems~\cite{stallman11}. 
\begin{enumerate*}[label=(\arabic*)] 
\item The user interface subsystems, consists of several actual interfaces, plus their supporting code. 
\item The symbol handling subsystem, consists of object file readers, debugging info interpreters, 
symbol table management, source language expression parsing, and type and value printing. 
\item The target system handling consists of execution control, stack frame analysis, and 
physical target manipulation.
\end{enumerate*}

DeloreanJs~\cite{leger23} is a back-in-time debugger for JavaScript programs. The value of 
back-in-time debuggers is the possibility to rewinding execution to a specific point, allowing the 
possibility for developers to, for instance, test different variable values within the same execution 
context, to better understand errors or explore hypothetical scenarios in the program's execution 
evolution~\cite{hofer06,lienhard09}. DeloreanJs has three main features: the ability to navigate 
through an execution history, modify values of a variable, and resume execution from a timepoint (at 
specific execution points in the past)~\cite{leger23}. DeloreanJS extends continuations with static 
analysis functionalities to capture the current program state and store the (history of) variable's 
values. There are several other debuggers for Javascript, most of them based on the use of 
breakpoints, in which variables cannot be modified to resume current execution with the new values, 
similar to the behavior offered in \ac{GDB}.

We note the functionality to go back in time, observe the program's state at a previous moment and 
execute a new timeline with the new values, offered by DeloreanJS, and in general back-in-time 
debuggers, is desirable in the context of \ac{RL} programs. Therefore, we use  back-in-time 
debuggers, and DeloreanJS, as a reference to create our debugger for \ac{RL} programs.


%%
\subsection{Python debuggers}
\label{sec:py}

Given that  most \ac{RL} programs are written in Python, we pay special attention to existing 
debuggers for Python. There are several Python debuggers available. We focus on those debuggers
that have support for the features required to debug \ac{RL} programs, such as the ability to navigate 
through an execution history (like a time travel debugger), and the ability to modify the values
of variables and resume the execution from a specific point in time.

% PDB
The \ac{PDB}~\cite{python-pdb} debugger is the built-in Python Debugger in CPython~\cite{shaw21} 
(the standard Python bytecode interpreter), implemented by several IDEs that provide a missing 
visual interface. \ac{PDB} is a powerful interactive source code debugger for Python programs, 
implemented as a module that can be used in Python scripts. \ac{PDB} provides a command-line 
interface for debugging Python programs, allows developers to define breakpoints, step through 
program instructions, inspect variables, and evaluate expressions. Additionally, \ac{PDB} provides 
a post-mortem debugging feature that allows developers to debug a program after it has crashed. 
However, \ac{PDB} does not have the ability to navigate through the execution history of a program, 
modify values of variables, or resume execution from a specific point in time, which are the features 
we are looking for in our solution. 

% PuDB
PuDB~\cite{pudb} is a full-screen, console-based visual debugger for Python. PuDB allows 
developers to set breakpoints, step through code, inspect variables, and evaluate expressions. 
Similar to \ac{PDB}, PuDB also provides a post-mortem debugging feature that allows developers to 
debug a program after it has crashed. However, PuDB does not have the ability to navigate through 
a program's execution history, modify values of variables, or resume execution from a specific point 
in time. Nevertheless, the history navigation feature can be simulated placing breakpoints in the 
code and restarting the execution until the breakpoint is reached. The UI makes it possible to go 
through the code and see the variables and the stack trace in a user-friendly fashoin.

% PyTrace
PyTrace~\cite{pytrace} is a time travel recorder/analyzer for Python. PyTrace records code execution, 
variables and stack frames. It has a UI that allows developers to navigate through the code, 
see the variables and the stack trace, and it is very user-friendly. However, PyTrace is not 
a full-featured debugger, as it does not allow you to change the values of variables, or do 
any other operation available to debuggers. PyTrace works as a tool to understand the behavior of 
a program, moving backward and forward through a recorded execution (again in a single execution 
path),  with different  graphs to see the state of your variables at a moment in time.

% RevPDB
\ac{RevPDB}~\cite{revdeb} is a reverse debugger for Python programs, RevPDB is an extension of 
\ac{PDB} built on top of the PyPy interpreter, allowing developers to go forward and backward in time. 
However, RevPDB is not a full-featured debugger, nor it has a user-friendly UI, as it is a proof of 
concept of the time traveling features, and it is not currently maintained.


% UDB
\ac{UDB}~\cite{udb} is a proprietary debugger, developed by the Company Undo, for Python 
programs that allows developers to navigate through time and inspect the state of the program at a 
previous moment in time. \ac{UDB} works on the principle of first recording 
a program while it is running normally and then replaying the execution while allowing the 
developer to navigate and inspect the program's state at different points in time. \ac{UDB} works at 
the process level, rewinding and replaying the state of the entire process. Technically, 
\ac{UDB} does allow new code paths to be executed at replay-time, but all execution replays are 
isolated.

\fref{tab:python-debuggers} shows a comparative summary of existing Python debuggers, and the 
features they offer. 

\begin{table}[hptb]
  \centering
  \caption{Comparative summary of existing Python debuggers.}
  \resizebox{\columnwidth}{!}{%
\begin{tabular}{c|lllll}
\textbf{Features}               & \multicolumn{1}{c}{\textbf{PDB}}                                                         & \multicolumn{1}{c}{\textbf{PuDB}}                                                                               & \multicolumn{1}{c}{\textbf{PyTrace}}                                           & \multicolumn{1}{c}{\textbf{RevPDB}}                                                   & \multicolumn{1}{c}{\textbf{UDB}}                                                                                                          \\ \toprule
\textbf{Description}            & \begin{tabular}[c]{@{}l@{}}Terminal debugger based\\ on CPython interpreter\end{tabular} & \begin{tabular}[c]{@{}l@{}}UI Debugger that extends\\ PDB debugger\end{tabular}                                 & Post mortem debugger                                                           & \begin{tabular}[c]{@{}l@{}}Terminal debugger based\\ on PyPy interpreter\end{tabular} & \begin{tabular}[c]{@{}l@{}}Not open source.\\ Has functionalities for \\  back in time debugging\\ for pdb.\end{tabular} \\
\textbf{Stepping}               & It allows stepping                                                                       & It allows stepping                                                                                              & It allows stepping                                                             & It allows stepping                                                                    & It allows stepping                                                                                                                        \\
\textbf{Breakpoints}            & It allows breakpoints                                                                    & It allows breakpoints                                                                                           & Does not allow breakpoints                                                      &                                                                                       & It allows breakpoints                                                                                                                     \\
\textbf{Stepping back}          & \begin{tabular}[c]{@{}l@{}}It can be simulated using \\ jumps through time\end{tabular}  & \begin{tabular}[c]{@{}l@{}}It can be simulated restarting\\ the execution and using \\ breakpoints\end{tabular} & It allows stepping back                                                        & It allows stepping back                                                               & It allows stepping back                                                                                                                   \\
\textbf{Variables modification} & \begin{tabular}[c]{@{}l@{}}It allows variables\\ modification\end{tabular}               & \begin{tabular}[c]{@{}l@{}}It allows variables\\ modification\end{tabular}                                      & \begin{tabular}[c]{@{}l@{}}Doesn't allow variables\\ modification\end{tabular} & \begin{tabular}[c]{@{}l@{}}It allows variables\\ modification\end{tabular}            & \begin{tabular}[c]{@{}l@{}}It allows variables\\ modification\end{tabular}                                                                \\
\textbf{GUI}                     
  & Implemented by the IDE
  & Does have a GUI                                                                                                  
  & Does have a GUI                                                                   
  & \begin{tabular}[c]{@{}l@{}}In it's simple form does \\ not have a GUI\end{tabular}     
  & Does have a GUI                                                                                                                           
\end{tabular}%
}
  \label{tab:python-debuggers}
\end{table}


%%
\subsection{Debuggers for \ac{AI} programs}
\label{sec:ai}

We now discuss the debuggers or debugger-like tools available for \ac{AI} and \ac{ML} programs. 
Identifying the root causes of bugs in machine learning systems is a complex task. Some bugs may 
originate within the models themselves, and their black-box nature makes pinpointing faults especially 
difficult. Additionally, the code used to train or run these models might be flawed, adding another layer 
of complexity. In systems that rely on \ac{ML} pipelines, bugs may not reside in the individual 
components but rather in how these components interact. Hardware miss-configurations are also a 
potential source of faults. Unfortunately, traditional debuggers embedded in IDEs are limited when 
detecting \ac{ML}-specific bugs. Therefore, there is an increasing interest on techniques and tools for 
debugging and testing \ac{ML} models.  

\ac{ML} programs are often debugged using visualization tools that allow developers to observe 
the program's behavior and analyze their internal state. The advance on \ac{ML} debugging tools 
represent an important step in towards  visualization and debugging tools for \ac{RL} programs.
Most of the existing tools are postmortem, not allowing developers to interact with the 
program during execution; therefore, making the tools unsuitable to analyze \ac{RL} programs and 
the continuous interaction between the environment and the agent.

CockPit~\cite{schneider21} is an open source debugging tool designed for deep neural network 
training, enabling practitioners to observe internal training dynamics in real time —much like 
traditional debuggers, it provides insights into program execution. Rather than relying solely on loss 
and accuracy curves, Cockpit offers a suite of instruments that monitor higher-order metrics such as 
gradient statistics, curvature, and sharpness, which are now efficiently computable during training. 
These insights help identify algorithmic failure modes—like unsuitable learning rates or optimization 
plateaus—and support more informed adjustments. Available as an open-source PyTorch package, 
Cockpit enhances both the interpretability and explainability of deep learning processes by revealing 
previously opaque aspects of training behavior.

TensorFlow offers several tools for debugging tensorFlow programs such as TensorFlow Debugger 
and TensorBoard Debugger~\cite{tensorboard}. The former allows for interactive execution of 
TensorFlow graphs, setting breakpoints, and inspecting tensor values in real-time.  The debugger 
can also be used in monitoring mode to  logs various aspects (at runtime) of a tensor flow program 
by instrumenting it with API calls.  The debugger can be combined with the TensorBoard to have 
visualizations/inspectors of the program state such as alerts, python execution timeline, graph 
executions, stack trace, and a source code viewers. However, TensorBoard does not allow 
developers to interact with the model during execution, which is essential for debugging \ac{RL}.

\ac{Vizarel}~\cite{deshpande20} is a debugger for \ac{RL}. \ac{Vizarel} implements an interactive 
visualization tool for debugging and interpreting \ac{RL} 
programs. The system offers different views that encapsulate the spatial and temporal dimensions 
of agent policies. The tool consists of a set of \emph{viewport modules}, each of which is an 
abstract entity that can be backed by different specs, conditioned on the underlying 
data~\cite{deshpande20}. This tool is useful to understand the behavior of an agent. However, it 
does not allow the user to interact with the variables during execution, which leads to a longer training 
process, having to wait until the end of the training run to know whether the agent is learning correctly 
or not.

\citet{rajan23} present a tool called \ac{MDP} Playground. This platform is intended to help 
developers and researchers to understand \ac{RL} agents better on toy complex environments, and 
to create unit tests characterizing agents' behavior  on toy \ac{MDP} examples, based on the 
OpenGym library. The problem addressed by \ac{MDP} Playground is the fragility of \ac{RL} 
programs, especially with the growth of deep \ac{RL} research growing. The complexity of the 
dynamics of \ac{RL} programs increases, leading to difficulties in identifying states or rewards 
distribution; if the agent's policy is not as expected, debugging \ac{RL} programs becomes 
hard~\cite{rajan23}. The idea with \ac{MDP} Playground is to open the programs, making their 
internal state available, to understand the behavior of the agent. Based only on toy examples, 
\ac{MDP} Playground is not scalable, nor it is suitable for real-world applications, but a 
proof-of-concept good to understand the behavior of \ac{RL} agents. While the \ac{MDP} Playground 
is not a debugger, it can help to debug the program by understanding the behavior of the agent in a 
controlled environment.

Finally, another recent work to debug \ac{RL} programs, sets as its purpose to gain trust in \ac{AI} 
programs using systematic testing~\cite{steinmetz21}. Erroneous program behavior (\ie bugs) lead to 
a bad agent policy, for example, not reaching the goal, or not learning the correct policy. The purpose 
of the work is to provide a systematic testing approach to find bugs. As part of the process, the tool 
offers a methodology to find potential bugs, or confirm a bug exists, using the Framework Fuzzing.

While the approaches presented are interesting to help understand \ac{RL} programs, none of them 
are actual debuggers, leaving a gap in the current literature, which is addressed as the main focus of 
our work. Our work proposes a debugger that allows the user to navigate through the execution 
history, modify the values of variables, and resume the execution from a specific point in time, 
performs into creating better \ac{RL} programs.
Open source and commercial tools like CockPit, DeepKit, TensorFlow Debugger, Amazon SageMaker 
Debugger,  TensorWatch, and Neptune.ai offer different features to debug/monitor ML systems. 
However, to the best of our knowledge, there is no open source option to debug  
\ac{RL} programs. 


\endinput

 



% $Id: solution.tex 
% !TEX root = ../main.tex

\section{Flik: A Bugs Life Debugger}
\label{sec:solution}

The black box behavior of \ac{ML} and \ac{RL} agents posit a problem to understand agents and 
their behavior, specially if unexpected behavior is observed. We posit debuggers as an appropriate 
tool to understand agents' behavior. Moreover, we posit debuggers with the possibility to modify 
programs' state and continue execution using new execution paths are even more appropriate and 
suitable to understand \ac{RL} agents, given their intrinsic nature of continuous interaction with the 
environment. However, currently, there is no debugger for \ac{RL} programs with the desired 
features. The reason for this is that most of the debuggers are postmortem, or do not allow developers
to go back in time to replay and analyze the program state. Additionally, most of the debuggers do not 
allow developers to modify variables. In order to contribute to the development of \ac{RL} programs, 
debugging tools should exhibit the following features:

\begin{description}
    \item[Stepping back:] Due to the execution loop of \ac{RL} programs we 
    want to have a functionality that will allow us to step back the execution to observe the change 
    in state between iterations of the loop. This will let developers interacting directly with the program, 
    without having to stop the execution,  lose the program state, or the training data already 
    accumulated. Such feature will help in identifying the root cause of erroneous agent behavior, 
    whether that is an error in the program's design for particular interactions, or an ill-defined 
    hyperparameter. 
    \item [Modifying variables:] While stepping back into the program's execution, we would want to be 
    able to modify variables' values during the execution. Such feature would help to test out the 
    behavior of an agent with different state-values, or hyperparameters, without having to 
    continuously stop and retrain the agent, which can be very costly and time-consuming. 
\end{description}

These features aim at tackling the problem of interaction with the program, as it allows 
to inspect the behavior of a program with respect to different variable values. Additionally, 
developers may interact with the program execution by going forward and backward, observing the 
effects of specific interactions between the agent and the environment.

Therefore, the proposed solution in this work is the creation of a back-in-time debugger with the 
functionalities mentioned above, as they provide more value to debug \ac{RL} agents. This 
is not a traditional debugger; rather, it allows for an understanding of the 
internal state of the agent, the decisions it makes, and the rewards it 
receives. In other words, it helps the developer to understand the execution context of the
agent in terms of variables, values, environment, states and the rewards. Additionally, with 
our debugger, developers can interact with the program at run time, modifying the values 
of the variables and continue the program's execution.

Moreover, we aim to provide a deeper understanding of the behavior of 
RL programs and to identify errors that arise during the learning process. 
This would allow for the evaluation of the construction and quality of 
software developed for RL and help formulate strategies to improve the 
development of these programs.

Thus, this solution proposes a framework that enables developers to 
analyze RL programs, evaluate their behavior, and observe the evolution 
of different variables over time.

In this chapter, we present the design and implementation of the debugger tool: \flik.
The debugger is a key component of the proposed framework, as it allows 
developers to interact with the RL program during execution, inspect the 
internal state of the agent, and modify its behavior in real-time.

\flik is a console-based debugger, constructed over \ac{PDB} debugger. \flik adds features 
such as colored syntax highlighting, tracking of variable states, and capturing stdout output 
from executed lines. The following are the three major features \flik has:
\begin{itemize}
    \item \flik saves the state in each step, it saves the local and global variables, 
    in a history variable. Additionally, other metadata like the information of the line 
    being executed is saved. This allows us to work later on the stepping back feature. 
    \item Additionally, \flik knows how to restore a previous state of a program, as 
    in the previous feature saves every information needed from the stack and the variables 
    to restore the state based on the history. 
    \item Finally, \flik performs the stepping back feature, it is created as a custom 
    \ac{PDB} command, and it has the same format as any other command. This takes the state
    saved in the specific line of code you want to step back, and it restores the state according
    to that information. 
\end{itemize}

In Python, the internal state of a program during execution is primarily encapsulated in 
stack frames. Each stack frame contains information about the execution state of a function 
call, including the current line number, local and global variables, and other metadata:
\begin{itemize}
    \item $f\_lineno$: The current line number being executed.
    \item $f\_locals$: A dictionary of local variables within the frame.
    \item $f\_globals$: A dictionary of global variables accessible within the frame.
    \item $f\_code$: A code object representing the function's bytecode and source code metadata.
\end{itemize}
The state is saved in a list that stores snapshots of the frame's state (line and variables) 
at each step, which allows stepping back into the state we want to restore. The exec function can execute 
in a list that stores snapshots of the frame's state (line and locals) at each stepcode in a 
specified frame's context by passing $f\_globals$ and $f\_locals$ as parameters. This allows 
\flik to simulate running a specific line in the context of a previous frame, maintaining 
both local and global variable references, essentially restoring the saved execution state.

This all done by extending the PDB class and adding custom commands to support stepping back
and a custom interface which allows the user to interact with the debugger. The interface 
displays the code, variables, and execution point, and allows the user (to use the \ac{PDB} 
functions) to pause, step forward, step back, continue or restart the program, as well as to 
modify and inspect variables. 

\fref{fig:debuggerf} presents the general interface of \flik. In 
the upper frame we have the running execution, in this case the print 
for the array to be sorted. The middle frame shows the source code that is being 
debugged. The lower frame shows the variables used in the 
program so far. Finally, in the bottom of the interface we have the interactive console in which the user can 
send the corresponding commands to the debugger.

\begin{figure}[h]
    \centering
    \includegraphics[width=1\textwidth]{figures/flik_interface.png}
    \caption{Debugger tool}
    \label{fig:debuggerf}
\end{figure}

% Add example of use step by step. Explain the videos.
Now, let's use an example to further understand how \flik works. Let's work with 
the usual gridworld example of \ac{RL} programs. The environment is a grid of 
$10 \times 10$ and the agent can move in four directions: up, down, left, and right.
There are two kind of rewards that the agent can get, a positive reward of 1 when 
the agent gets to the goal, and a negative reward of -1 when the agent goes to the 
trap. The agent starts in the blue box shown \fref{fig:gridworld}. Now, in the following
example let's say we define our program properly, but we define the $\epsilon$ value 
so small that the agent will not explore the grid, and will not learn properly. 
This will mean that only the first action taken by the agent will be random and after that
with a major probability it will keep only choosing the same action. We can think about 
the worst case scenario, in which the agent will move right and then in the next step it 
will move left, and it will keep doing this for a long time, with a low probability of 
choosing another action. We can use \flik to debug this problem, we can go step by step
inspecting the variables and the actions taken and we can identify specifically that for 
the first state in each episode the agent is taking the same action. We can then go back and
reproduce this problem as many times as we want, and we can change the value of $\epsilon$
to a higher value, so the agent can explore the grid properly. And finish the execution.
This example is shown in the following video: \url{https://drive.google.com/file/d/1NyipuWsRr6ZrIbtlvU5qyooHS2aVsWXc/view?usp=sharing}.



\endinput


% $Id: evaluation.tex 
% !TEX root = ../main.tex

\section{Empirical Study}
\label{sec:evaluation}

This section presents the empirical study for \flik. The purpose of the empirical study assessing the 
user experience and usefulness of \flik, when used to identify and locate bugs in  \ac{RL} programs. 
In particular, the participants were provided with three different \ac{RL} programs that implement three 
\ac{RL} problems. The problems were chosen according to their familiarity and an increasing 
environment complexity. 
\begin{enumerate*}[label=(\arabic*)]
\item GridWorld, is a standard benchmark for \ac{RL} based on episodic learning.
\item Rooms, is common \ac{RL} problem used as an example of the complexity of environments and 
the possibility por agents to abstract their knowledge in hierarchical learning~\cite{pateria21} 
structures. \item Finally, Driving assistant is a more recent \ac{RL} environment, uncommon to most 
developers, used to learn different tasks for autonomous driving in a continuous environment. 
\end{enumerate*}

A fault was manually injected in each program, and the task assigned to the participants was to improve the functional quality of the three programs. Note that the injected bugs did not lead the program to exceptions or crashes, i.e., each program ran and there was no evident buggy behavior. The injected bugs made the programs to generate outputs that do not match the expected outputs. Concerning the participants, master students from the Reinforcement Learning course at Universidad de los Andes were recruited.  The participants were provided with a evaluation  \fref{sec:eval-guide}, that describes the purpose of the tool, the task and steps, instructions on how to run and 
use the tool, documentation of the tool, and an small example. The real-world problems implemented for the study are described in the following:

%%%%
\paragraph{\textbf{GridWorld.}}
This environment consists of a $n \times n$ ($10\times 10$ in our example) 
rectangular  board/grid, in which each tile $(i,j)$ represents a specific state of the board. Tiles in the board may be  walls, which agents cannot cross. Additionally, there are special exit  tiles that give a positive or negative reward to agents, as shown in \fref{fig:gridworld}. All tile types are unknown to the agent that moves from a given starting point in the board, searching for the goal state (\ie exit states with positive reward of $1$). The agent moves from state to state, avoiding  obstacles and incorrect exit states (which give a reward of $-1$ when used to exit). 

The program has a bug for the $\epsilon$ parameter, presenting an error in the probability which defines which action is taken next. This introduces a wrong behavior for the \ac{RL} 
agent because $\epsilon$ will be so small that only in the first iteration the action will be taken randomly and in the follow iterations the probability will be so small to take another action that the agent will not explore other action policies. This is an undesired behavior specially for training purposes as an agent should have a large probability to choose other actions and explore the grid. The idea in this task is that the study participants use \flik to navigate through the code and find out why the agent is not learning properly. And eventually, the participants should come out with the solution of increasing the value of $\epsilon$. 

\begin{figure}[hptb]
  \centering
  \includegraphics[width=0.5\columnwidth]{figures/gridworld.png}
  \caption{10x10 gridworld environment example}
  \label{fig:gridworld}
\end{figure}

%%%%
\paragraph{\textbf{Rooms.}} 
The four rooms maze environment consists of a $13\times 13$ board/grid divided in $4$ sections 
(\ie rooms), with walls between them, and a door opening to go from one room to another, as shown 
in \fref{fig:rooms}. The agent's objective in this environment is to exit through the upper-left room 
(the green square) in the fewest possible steps. Reaching the exit state gives a reward of $1$, and no 
other action give a reward to the agent. In each episode the agent starts from any valid position in the 
grid, \eg the yellow square in the bottom-right room in the figure. 

\begin{figure}[hptb]
  \centering
  \includegraphics[width=0.5\columnwidth]{figures/rooms.png}
  \caption{Rooms environment example with associated rewards in each state}
  \label{fig:rooms}
\end{figure}

In the program, we introduced a bug in the learning rate $\alpha$, on the Q-learning equation, thus,  the original values for the learning rate alpha are exchanged like:

\begin{itemize}
\item Original equation:
$
Q(s, a) \leftarrow (1-\alpha) Q(s, a) + \alpha \left( r + \gamma \max_{a'} Q(s', a') \right)
$

\item Erroneous equation to debug:
$
Q(s, a) \leftarrow  \alpha Q(s, a) + (1-\alpha) \left( r + \gamma \max_{a'} Q(s', a') \right)
$
\end{itemize}

In the original equation,  $(1-\alpha) Q(s, a)$ is the current value and $\gamma \max_{a'} Q(s', a')$ 
the maximum reward that can be obtained from state $s'$. This means that if the learning rate is very 
small the current value will keep almost the same, turning a little bit towards the reward and the 
maximum value given the action; this will make that the agent learn short steps towards the 
optimal policy. In the Q-learning formula, the learning rate $\alpha$ defines how much the old estimate $Q(s,a)$ 
is revised based on the new information. It ensures that over time, the algorithm balances past 
knowledge with current learning, gradually incorporating new information while retaining important 
aspects of previous learning. Changing the equation in this way will disrupt this balance. Specifically,
$(1-\alpha)$ scales the difference between the new estimate and the old estimate. This makes the new information less influential as $\alpha$ gets larger, while the old value gets re-scaled by $\alpha$,  which does not align with the expected behavior of a Q-learning update. The idea in this task is that the participants use \flik to navigate through the code and find out the reason why the agent is not learning properly, afterwards we expected the participant to figure out a solution adjusting the proper value to update the Q-Learning equation.

%\subsection{Exercise 3: Driving Assistant}
%\label{sec:cars-eval}
\paragraph{\textbf{Cars (Driving Assistant).}} In this case the agent must learn to drive, on the correct lane, at the allowed speed, taking over slow traffic, and not crashing. The possible actions for the agent are: straight, slow\_down, speed\_up, steer\_left, steer\_right. Figure \ref{fig:cars-code-example} depicts the visual interface of this environment.

\begin{figure}[h]
    \centering
    \includegraphics[width=0.5\textwidth]{figures/cars_example.png}
    \caption{Cars code example}
    \label{fig:cars-code-example}
\end{figure}

This is the largest task for the participants to analyze with \flik. % and see why the agent was not learningcorrectly, with respect to the expected. 
The bug introduced in this application is in the reward function, not motivating the agent to drive at the speed limit. The policy learned by the agent emerges as stopping or go very slow, which is not the expected behavior. Through exploration using \flik, developers are expected to observe the behavior of the reward and update it so that the agent can learn to drive appropriately.
as expected.

%% Add way of evaluation, explanation about the survey, questions.
%\subsection{Evaluation setup}
%\label{sec:evaluation}

For the user evaluation we had a session with 27 developers (\ie graduate students). 
The developers were asked to complete the three tasks (\ie gridworld, rooms, cars) described previously. The full evaluation session used a time-slot of 1h30 split as follows 
\begin{enumerate}[label=(\arabic*)]
\item During the first 15 minutes the tool and a usage example were presented to the participants; (shown in \fref{sec:eval-guide}).
\item 50 minutes to complete the tasks; 15 minutes to finish the first task, 15 minutes to finish the second task, and  20 minutes to finish the third task. 
\item At the end of the session, the students were asked to fill an evaluation survey 
\end{enumerate}

The evaluation survey, defined as a google form, is divided in three sections: general knowledge questions, task questions, and usability.
The general knowledge questions were about the participants' experience  using Python, debuggers and  terminal. The task questions were about the bug encountered in each of the tasks, and how the the bug was fixed (if fixed). The usability  questions were regarding the participants experience with the debugger usability, and additional feedback they could provide for improving \flik. The survey was anonymized, and the questions were inspired by the 
back in time debugger for JavaScript study~\cite{leger23}, and the responses are available 
at \url{https://shorturl.at/DhN56}. 

The survey had 23 multiple choice questions with a 5-points Likert scale,  in which 5 meant completely agreed, and 1 meant completely disagreed. One of the questions was a yes or no answer, to identify if the students wanted to use the tool in the future. Finally, there were 10 open-answer questions , to dive deeper into feedback for the tool, and the tasks' complexity.


%Finally, there are two examples of the tool being used. In the first  example, the tool is used to debug a simple program, this was used to introduced to the students  the simple commands they could use like stepping (forward and backwards), modifying, or inspecting variables. In the second example, the tool is used to debug a \ac{RL} program, specially the gridworld example. The link to find this videos is: \url{https://shorturl.at/rD343} 



\endinput


% $Id: results.tex 
% !TEX root = ../main.tex

\section{Analysis and Results}
\label{sec:results}

This section presents the analysis of the results of the survey presented in \fref{sec:evaluation}. 
This results are presented according to three sections of the survey: General knowledge (\fref{sec:general-knowledge}),
Task Effectiveness (\fref{sec:tasks-results}) and usability (\fref{sec:usability}). Additionally, a 
discussion section is added analyzing the results form the survey. All additional data, not mentioned 
on this section, is given in Appendix A.2.

\section{General Knowledge Results}
\label{sec:general-knowledge}
Most of the participants were very experienced on the use of python as it
can be seen in (\fref{fig:general-know} (a)), most of the people had a high level of expertise (over 5)
as they used Python very often in their work and university. 
Most of the people were very familiarized using \ac{RL} algorithms (\fref{fig:general-know} (b)), as they were 
taking a course on the topic. This is very important information, as the bugs required familiarity with \ac{RL} and python 
knowledge to be identified. Furthermore, the students were very familiarized with 
the terminal, this means it would be easy for them to interact with the tool, as the commands 
and the interface require experience using the console (\fref{fig:general-know} (c)). Nevertheless, 
average the 
students weren't familiarized with debuggers, and they only used them very rarely, or used 
Visual Studio Code interface simple features, like breakpoint function. This lead to 
misunderstandings and probably difficulties in making the tasks, as the tool is based on
previous debugger (\fref{fig:general-know} (d)). Also, this made the learning curve of the tool 
steeper than with a person with previous debugger experience.

Note that in the figures in \fref{fig:general-know} the y-axis represents the number of people 
that answered the question with that score, and the x-axis represents score, which could have a 
possible value of 5 being completely agreed, and 1 being completely disagreed.

\begin{figure}
    \centering
    \subfloat[]{\includegraphics[width=0.25\textwidth]{figures/experience-python.png}} 
    \subfloat[]{\includegraphics[width=0.25\textwidth]{figures/experience-rl.png}} \\
    \subfloat[]{\includegraphics[width=0.25\textwidth]{figures/experience-terminal.png}}
    \subfloat[]{\includegraphics[width=0.25\textwidth]{figures/experience-debuggers.png}}
    \caption{(a) Python Experience (b) RL Experience (c) Terminal Experience (d) Debuggers Experience}
    \label{fig:general-know}
\end{figure}



\section{Tasks Results}
\label{sec:tasks-results}
As there can be seen in the tables \fref{tab:grid-results}, \fref{tab:rooms-results} and 
\fref{tab:cars-results}, most of the people made it through the first task (\fref{fig:task1}),
they thought it was an easy task to solve. This task besides being easy, took them a lot of 
time to complete because this was the task in which they were getting familiarized with the 
tool. The second task was harder than the first one, most of the people took longer to finish the 
task (\fref{fig:task2}), and most of the people had trouble finding the problem to the program. 
This was meant to be as task 2, wasn't meant to be completely wrong, it was just taking huge 
steps, and the error wasn't because a variable definition, but because a changed in the Q-learning
algorithm, which made a lot of students initially confused. Finally, the third task (\fref{fig:task3}) was the hardest 
one, most people had trouble finding the bug. They thought it wasn't an easy task to solve, 
nevertheless, most of the people managed to finish the task in less time than in the other tasks. 
In this case, it is understandable that for the first task besides being 
easy people took longer, because they were getting familiarized with the tool. For the second task,
given the nature of the bug and the fact that the bug was introduced in the Q-learning algorithm, 
it was harder to find the bug, and for that people thought it was harder to solve, but
most of the people finish it. Finally, for the third task, the bug was introduced in the rewards,
people took less as they had more practice with the tool.

Note that in the figures in \fref{fig:general-know} the y-axis represents the number of people 
that answered the question with that score, and the x-axis represents score, which could have a 
possible value of 5 being completely agreed, and 1 being completely disagreed. Except for the 
last question, in which 5 was taking a lot of time, and 1 was taking very little time.

\begin{figure}
    \centering
    \subfloat[]{\includegraphics[width=0.32\textwidth]{figures/task1.png}} 
    \subfloat[]{\includegraphics[width=0.32\textwidth]{figures/task2.png}} 
    \subfloat[]{\includegraphics[width=0.32\textwidth]{figures/task3.png}}
    \caption{(a) GridWrold (b) Rooms (c) Driving Assistant}
    \label{fig:general-know}
\end{figure}

\section{Debugger Usability Results}
\label{sec:usability}

Regarding the debugger usability (\fref{tab:general1-debuggers} and \fref{tab:general2-debuggers}), it can 
be seen that the tool was useful,
and the general comments were that the tool would be useful if there was more time to study it. 
And to learn about the tool before using it in real tasks. The 
tool was easy to use, once they got to know it a little better, and they got to practice more with the commands.
Additionally, there were several comments about the tool improving the UI,
as it was hard to understand the tool at first sight, it wasn't similar to the Visual Studio Code 
interface that most of the people were used to use.

Note that in the figures in \fref{fig:general-know} the y-axis represents the number of people 
that answered the question with that score, and the x-axis represents score, which could have a 
possible value of 5 being completely agreed, and 1 being completely disagreed.

\begin{table}[H]
\centering
\resizebox{\columnwidth}{!}{%
\begin{tabular}{
>{\columncolor[HTML]{FFFFFF}}c 
>{\columncolor[HTML]{FFFFFF}}c 
>{\columncolor[HTML]{FFFFFF}}c 
>{\columncolor[HTML]{FFFFFF}}c 
>{\columncolor[HTML]{FFFFFF}}c 
>{\columncolor[HTML]{FFFFFF}}c }
% \multicolumn{6}{c}{\cellcolor[HTML]{FFFFFF}{\color[HTML]{383838} \textbf{Task 2: Rooms Experiment.}}}                                                                                                                                                                                                                                                                                                                                                                                                                                                                                                                                                                                                 \\
{\color[HTML]{383838} \textbf{}}  & {\color[HTML]{383838} \textbf{\begin{tabular}[c]{@{}c@{}}I have found too much \\ inconsistency in this system\end{tabular}}} & {\color[HTML]{383838} \textbf{\begin{tabular}[c]{@{}c@{}}I think most people would learn \\ to use the system quickly\end{tabular}}} & {\color[HTML]{383838} \textbf{\begin{tabular}[c]{@{}c@{}}I found the system quite \\ awkward to use\end{tabular}}} & {\color[HTML]{383838} \textbf{\begin{tabular}[c]{@{}c@{}}I have felt very safe \\ using the system\end{tabular}}} & {\color[HTML]{383838} \textbf{\begin{tabular}[c]{@{}c@{}}I would need to learn a lot \\ of things before I could handle the system\end{tabular}}} \\
{\color[HTML]{383838} \textbf{5}} & {\color[HTML]{383838} 1}                                                                                                      & {\color[HTML]{383838} 7}                                                                                                             & {\color[HTML]{383838} 4}                                                                                           & {\color[HTML]{383838} 7}                                                                                          & {\color[HTML]{383838} 8.0}                                                                                                                        \\
{\color[HTML]{383838} \textbf{4}} & {\color[HTML]{383838} 2}                                                                                                      & {\color[HTML]{383838} 6}                                                                                                             & {\color[HTML]{383838} 7}                                                                                           & {\color[HTML]{383838} 8}                                                                                          & {\color[HTML]{383838} 6.0}                                                                                                                        \\
{\color[HTML]{383838} \textbf{3}} & {\color[HTML]{383838} 4}                                                                                                      & {\color[HTML]{383838} 6}                                                                                                             & {\color[HTML]{383838} 9}                                                                                           & {\color[HTML]{383838} 7}                                                                                          & {\color[HTML]{383838} 4.0}                                                                                                                        \\
{\color[HTML]{383838} \textbf{2}} & {\color[HTML]{383838} 7}                                                                                                      & {\color[HTML]{383838} 6}                                                                                                             & {\color[HTML]{383838} 4}                                                                                           & {\color[HTML]{383838} 4}                                                                                          & {\color[HTML]{383838} 0.0}                                                                                                                        \\
{\color[HTML]{383838} \textbf{1}} & {\color[HTML]{383838} 13}                                                                                                     & {\color[HTML]{383838} 2}                                                                                                             & {\color[HTML]{383838} 3}                                                                                           & {\color[HTML]{383838} 1}                                                                                          & {\color[HTML]{383838} 9.0}                                                                                                                       
\end{tabular}%
}
\caption{General Results Part 1}
\label{tab:general1-debuggers}
\end{table}

\begin{table}[H]
\centering
\resizebox{\columnwidth}{!}{%
\begin{tabular}{
>{\columncolor[HTML]{FFFFFF}}c 
>{\columncolor[HTML]{FFFFFF}}c 
>{\columncolor[HTML]{FFFFFF}}c 
>{\columncolor[HTML]{FFFFFF}}c 
>{\columncolor[HTML]{FFFFFF}}c 
>{\columncolor[HTML]{FFFFFF}}c }
% \multicolumn{6}{c}{\cellcolor[HTML]{FFFFFF}{\color[HTML]{383838} \textbf{Task 2: Rooms Experiment.}}}                                                                                                                                                                                                                                                                                                                                                                                                                                                                                                                                                                                      \\
{\color[HTML]{383838} \textbf{}}  & {\color[HTML]{383838} \textbf{\begin{tabular}[c]{@{}c@{}}I think I would like to use this \\ system frequently\end{tabular}}} & {\color[HTML]{383838} \textbf{\begin{tabular}[c]{@{}c@{}}I find this system \\ unnecessarily complex\end{tabular}}} & {\color[HTML]{383838} \textbf{\begin{tabular}[c]{@{}c@{}}I think the system \\ is easy to use\end{tabular}}} & {\color[HTML]{383838} \textbf{\begin{tabular}[c]{@{}c@{}}I think I would need technical \\ support to use the system\end{tabular}}} & {\color[HTML]{383838} \textbf{\begin{tabular}[c]{@{}c@{}}I find the various functions of the \\ system quite well integrated\end{tabular}}} \\
{\color[HTML]{383838} \textbf{5}} & {\color[HTML]{383838} 7}                                                                                                      & {\color[HTML]{383838} 2}                                                                                            & {\color[HTML]{383838} 5.0}                                                                                   & {\color[HTML]{383838} 9}                                                                                                            & {\color[HTML]{383838} 6}                                                                                                                    \\
{\color[HTML]{383838} \textbf{4}} & {\color[HTML]{383838} 5}                                                                                                      & {\color[HTML]{383838} 7}                                                                                            & {\color[HTML]{383838} 5.0}                                                                                   & {\color[HTML]{383838} 8}                                                                                                            & {\color[HTML]{383838} 13}                                                                                                                   \\
{\color[HTML]{383838} \textbf{3}} & {\color[HTML]{383838} 7}                                                                                                      & {\color[HTML]{383838} 7}                                                                                            & {\color[HTML]{383838} 11.0}                                                                                  & {\color[HTML]{383838} 4}                                                                                                            & {\color[HTML]{383838} 5}                                                                                                                    \\
{\color[HTML]{383838} \textbf{2}} & {\color[HTML]{383838} 7}                                                                                                      & {\color[HTML]{383838} 6}                                                                                            & {\color[HTML]{383838} 6.0}                                                                                   & {\color[HTML]{383838} 3}                                                                                                            & {\color[HTML]{383838} 2}                                                                                                                    \\
{\color[HTML]{383838} \textbf{1}} & {\color[HTML]{383838} 1}                                                                                                      & {\color[HTML]{383838} 5}                                                                                            & {\color[HTML]{383838} 0.0}                                                                                   & {\color[HTML]{383838} 3}                                                                                                            & {\color[HTML]{383838} 1}                                                                                                                   
\end{tabular}%
}
\caption{General Results Part 2}
\label{tab:general2-debuggers}
\end{table}

% \begin{table}[]
% \centering
% \resizebox{\columnwidth}{!}{%
% \begin{tabular}{
% >{\columncolor[HTML]{FFFFFF}}c 
% >{\columncolor[HTML]{FFFFFF}}c 
% >{\columncolor[HTML]{FFFFFF}}c 
% >{\columncolor[HTML]{FFFFFF}}c 
% >{\columncolor[HTML]{FFFFFF}}c 
% >{\columncolor[HTML]{FFFFFF}}c 
% >{\columncolor[HTML]{FFFFFF}}c 
% >{\columncolor[HTML]{FFFFFF}}c 
% >{\columncolor[HTML]{FFFFFF}}c 
% >{\columncolor[HTML]{FFFFFF}}c 
% >{\columncolor[HTML]{FFFFFF}}c }
% \multicolumn{11}{c}{\cellcolor[HTML]{FFFFFF}{\color[HTML]{383838} \textbf{Debugger Usability.}}}                                                                                                                                                                                                                                                                                                                                                                                                                                                                                                                                                                                                                                                                                                                                                                                                                                                                                                                                                                                                                                                                                                                                                                                                                                                         \\
% {\color[HTML]{383838} \textbf{}}  & {\color[HTML]{383838} \textbf{\begin{tabular}[c]{@{}c@{}}I think I would like to use this \\ system frequently\end{tabular}}} & {\color[HTML]{383838} \textbf{\begin{tabular}[c]{@{}c@{}}I find this system \\ unnecessarily complex\end{tabular}}} & {\color[HTML]{383838} \textbf{\begin{tabular}[c]{@{}c@{}}I think the system \\ is easy to use\end{tabular}}} & {\color[HTML]{383838} \textbf{\begin{tabular}[c]{@{}c@{}}I think I would need technical \\ support to use the system\end{tabular}}} & {\color[HTML]{383838} \textbf{\begin{tabular}[c]{@{}c@{}}I find the various functions of the \\ system quite well integrated\end{tabular}}} & {\color[HTML]{383838} \textbf{\begin{tabular}[c]{@{}c@{}}I have found too much \\ inconsistency in this system\end{tabular}}} & {\color[HTML]{383838} \textbf{\begin{tabular}[c]{@{}c@{}}I think most people would learn \\ to use the system quickly\end{tabular}}} & {\color[HTML]{383838} \textbf{\begin{tabular}[c]{@{}c@{}}I found the system quite \\ awkward to use\end{tabular}}} & {\color[HTML]{383838} \textbf{\begin{tabular}[c]{@{}c@{}}I have felt very safe \\ using the system\end{tabular}}} & {\color[HTML]{383838} \textbf{\begin{tabular}[c]{@{}c@{}}I would need to learn a lot \\ of things before I could handle the system\end{tabular}}} \\
% {\color[HTML]{383838} \textbf{5}} & {\color[HTML]{383838} 7}                                                                                                      & {\color[HTML]{383838} 2}                                                                                            & {\color[HTML]{383838} 5.0}                                                                                   & {\color[HTML]{383838} 9}                                                                                                            & {\color[HTML]{383838} 6}                                                                                                                    & {\color[HTML]{383838} 1}                                                                                                      & {\color[HTML]{383838} 7}                                                                                                             & {\color[HTML]{383838} 4}                                                                                           & {\color[HTML]{383838} 7}                                                                                          & {\color[HTML]{383838} 8.0}                                                                                                                        \\
% {\color[HTML]{383838} \textbf{2}} & {\color[HTML]{383838} 7}                                                                                                      & {\color[HTML]{383838} 6}                                                                                            & {\color[HTML]{383838} 6.0}                                                                                   & {\color[HTML]{383838} 3}                                                                                                            & {\color[HTML]{383838} 2}                                                                                                                    & {\color[HTML]{383838} 7}                                                                                                      & {\color[HTML]{383838} 6}                                                                                                             & {\color[HTML]{383838} 4}                                                                                           & {\color[HTML]{383838} 4}                                                                                          & {\color[HTML]{383838} 0.0}                                                                                                                        \\
% {\color[HTML]{383838} \textbf{3}} & {\color[HTML]{383838} 7}                                                                                                      & {\color[HTML]{383838} 7}                                                                                            & {\color[HTML]{383838} 11.0}                                                                                  & {\color[HTML]{383838} 4}                                                                                                            & {\color[HTML]{383838} 5}                                                                                                                    & {\color[HTML]{383838} 4}                                                                                                      & {\color[HTML]{383838} 6}                                                                                                             & {\color[HTML]{383838} 9}                                                                                           & {\color[HTML]{383838} 7}                                                                                          & {\color[HTML]{383838} 4.0}                                                                                                                        \\
% {\color[HTML]{383838} \textbf{4}} & {\color[HTML]{383838} 5}                                                                                                      & {\color[HTML]{383838} 7}                                                                                            & {\color[HTML]{383838} 5.0}                                                                                   & {\color[HTML]{383838} 8}                                                                                                            & {\color[HTML]{383838} 13}                                                                                                                   & {\color[HTML]{383838} 2}                                                                                                      & {\color[HTML]{383838} 6}                                                                                                             & {\color[HTML]{383838} 7}                                                                                           & {\color[HTML]{383838} 8}                                                                                          & {\color[HTML]{383838} 6.0}                                                                                                                        \\
% {\color[HTML]{383838} \textbf{1}} & {\color[HTML]{383838} 1}                                                                                                      & {\color[HTML]{383838} 5}                                                                                            & {\color[HTML]{383838} 0.0}                                                                                   & {\color[HTML]{383838} 3}                                                                                                            & {\color[HTML]{383838} 1}                                                                                                                    & {\color[HTML]{383838} 13}                                                                                                     & {\color[HTML]{383838} 2}                                                                                                             & {\color[HTML]{383838} 3}                                                                                           & {\color[HTML]{383838} 1}                                                                                          & {\color[HTML]{383838} 9.0}                                                                                                                       
% \end{tabular}%
% }
% \end{table}

\section{Discussion}
\label{sec:discussion}

In general, the results of the survey were very positive, most of the people thought that 
the tool was useful specially for the kind of challenges \ac{RL} programs could have. In 
spite of the tool being hard to familiarized with at the beginning, specially for the people who had very 
little experience with debuggers, most of the people thought that the tool was easy to use 
once you get to know it a little better. Additionally, people said that the tool would've 
been useful for the course development, as most of the main errors they had for their homework,
were related to the bugs being introduced in these tasks.

Additionally, it was expected that for the first task people used a bit more time to finish the 
task, besides being an easy task, as they were getting familiarized with the tool. For the second,
task, most of the people had trouble finding the bug, they needed to explore much more within the 
code before finding the bug, compared to the last task. Finally, it was expected that the developers
took less time to find the last bug, as they already had practice with the tool, and they were 
knew how to use it and how to interact with the program. Nevertheless, it was expected that 
the users felt the task harder than the previous one. It was also surprising that for this last task 
they found more bugs than the initially expected one.

\endinput


% $Id: conclusion.tex 
% !TEX root = ../main.tex

\section{Conclusion and Future Work}
\label{sec:conclusion}

In this work, we presented \flik, a back-in-time debugger for \ac{RL} programs.
We initially had two major problems: First, \ac{RL} programs were very complex in terms of the environment,
so the developer would have major problems in understanding the behavior of the agent, and
the possible bug being presented, in the program. Also the complexity of the program, requires
too much computational power which makes it even harder to interact to. This Lead us to our second problem,
the interaction between the agent and the environment is very hard to understand, and
for a developer that needs to debug the program, it is very hard to understand the behavior of the agent
and the interaction that a possible bug is causing.

A back in time debugger strategy showed being a really good strategy for developing and 
understanding RL programs. Most of the developers managed to understand the behavior of the agent,
and find the bug hidden on it. Specifically, \flik, a console-based debugger, demonstrated to 
be a very useful tool to find general bugs that an \ac{RL} program can present during development. 
The tool that allowed developers to interact with the program during 
execution, inspect the internal state of the agent, and modify its behavior in real-time. These were 
important features as we wanted \flik to be able to tackle the problem of interaction with the 
program to be able to inspect in a deeper way the state of the program in terms of variables, and 
it provides a functionality of interacting with the execution of the program going forward and backwards.
Additionally, \flik had a visual interface, which reduced the complexity of the \ac{RL} program,
making it easier for the user to inspect the state step by step.


As future work, on proposal form the student's feedback was to optimize memory consumption by 
using layer cashing in a similar was as git and docker way save states, so instead of saving the
history of the entire variables (globals and locals), and all the metadata of the program, we could 
only save the changes made on the program, this could optimize the memory consumption of \flik,
nevertheless, this would require a more complex logic to be implemented. 

Moreover, the students suggested that the visual interface could be improved, as it was not very 
intuitive for them to use it (taking into account most of them were not used to use a debugger).
\flik could have a VSCode plugging to be added to the editor, so the user could use the debugger 
in a more friendly way, and also have the possibility to use the debugger in a more visual way.

Finally, \flik could be improved by integrating the environment visuals made in matplotlib
graphs or pygame graphs in the visual interface, so the user could have a better understanding of 
the environment and the agent's behavior, while debugging the program.




% $Id: responses.tex 
% !TEX root = main.tex

\newpage

\appendix

\clearpage

\subsection{Response Letter}


As required by the reviewing process, we are submitting this response letter along with a revised 
version of our paper entitled \textsl{Flik: A Back-in-time Debugger for Reinforcement Learning Programs}. For convenience, the letter is formatted as an appendix of the paper.

We would like to thank our anonymous reviewers for their efforts in reviewing our submission and 
for their useful and detailed suggestions to improve it. We have considered all remarks carefully, and 
addressed them to the best of our possibilities.

There are \thetotalremarks~ different remarks for this iteration of the paper. Each remark 
is followed by our response and an explicit status indication:

\begin{itemize}[nosep,label=--] 
	\item \revtotal{solved} \statussolved 
	\item \revtotal{future} \statusfuture
	\item \revtotal{incorrect} \statusincorrect \footnote{We clarify the question made by the reviewer} 
	\item \revtotal{disagree} \statusdisagree\footnote{We ask the reviewer to reconsider the remark in the light of arguments and clarifications we provide in the response.} 
	\item \revtotal{unsolved} \statusunsolved  \footnote{We could not accommodate the remark in the text due to blocking constraints.} 
	\item \revtotal{unspecific} \statusunspecific\footnote{We answered the remark to our best interpretation. We are unsure about what the reviewer meant.} 	
\end{itemize}

The remarks with specific changes in the paper have a reference to the location in the paper 
where the solution is found. These solutions are marked with the 
\includegraphics[width=0.03\textwidth]{./figures/fix.pdf} icon in the paper margin for easy 
identification. Additionally, in the margin we point to the specific remark(s) the modification 
is addressing.


%%
\subsection{Review \#1}

\begin{remark} 
the presentation is quite poor. The paper contains a number of typos or odd sentences. 
\end{remark}
\begin{solved}
We did a full revision over the paper and cleaned up the errors suggested by the reviewer and other we found to improve the overall presentation
\end{solved}

\begin{remark} \label{rem:intro}
The introduction is overly technical.
\end{remark}
\begin{solved}
We rewrote the whole introduction to have a more abstract presentation of the motivation, and reduce its technicality (\fref[vario]{par:intro}). All the technical details have been moved to a new sections providing the background on \ac{RL} (\fref{sec:rl-background} - \fref[vario]{par:background})  
\end{solved}

\begin{remark} \label{rem:requirements}
Some ideas are repeated a lot, while other (relevant) ones are not mentioned at all. For example, it should be made clear from the beginning that Flik requires source programs and that it includes an instrumented interpreter to run programs, save states, recover a previous state, etc. (if this is the case).
\end{remark}
\begin{solved}
We present the requirements and execution characteristics in the introduction (\fref[vario]{par:requirements}), and then elaborate on the details of the requirements and inner working in the new  \fref{sec:flik-internals} (\fref[vario]{par:requirements2})
\end{solved}

\begin{remark}\label{sec:implementation}
the paper doesn't clarify how Flik was implemented. At the beginning, I thought it was a tool for runtime monitoring, since the authors mention in several places the idea of observing and, if needed, modifying a program execution. However, judging by the commands in the interface of the debugger, I assume it actually includes an instrumented interpreter to simulate the actual execution of Python code, or it makes calls to a Python interpreter, or... In any case, it's essential to describe how Flik is implemented in the paper.
\end{remark}
\begin{solved}
Aligned with \fref{rem:requirements}, we added \fref{sec:flik-internals} describing the implementation of \flik, in particular we: Added a figure with the design and details off \flik to help carry out the explanations.
\end{solved}

\begin{remark} 
- Page 2: mentioning that one disadvantage of debugging RL programs is that they're generally used as a black boxes made me think that Flik, in contrast, wouldn't need the source code. But I guess that's not the case. This argument is a bit confusing.
\end{remark}

\begin{remark}\label{rem:logs}
section 2.1: I missed some text about the use of logs in program debugging.
\end{remark}
\begin{solved}
The use of logs was added to the background section in \fref{sec:deb} (\fref[vario]{par:logs}).
\end{solved}

\begin{remark} 
there's also little mention of the techniques used in the debugger.
How exactly is the program state defined?
\end{remark}

\begin{remark}
What happens if there are I/O operations?
\end{remark}

\begin{remark}
Does it accept 100\% of Python? 
\end{remark}

\begin{remark}
what if there are calls to libraries whose source code isn't available?
\end{remark}

\begin{remark} \label{rem:scalability}
The experimental evaluation is interesting, but it doesn't say anything about the execution cost. If, as I mentioned earlier, Flik is defined as an instrumented interpreter, it is likely to have scalability issues, especially if it must save all the states of an execution (which can potentially be very long). It would be good to show execution times (e.g., of the complete execution of the program in Flik assuming there are no breakpoints) and compare it with a standard interpreter for the language. This way, the overhead introduced by the debugger could be analyzed.
\end{remark}
\begin{solved}
We added a discussion about the performance, overhead and scalability of using \flik according to the evaluation use cases in \fref{sec:scalability} (\fref[vario]{par:scalability})
\end{solved}


%%%%
\subsubsection{ MINOR COMMENTS}
\begin{remark}
solve bugs -> solveD bugs
\end{remark}
\begin{solved}
Updated as suggested by the reviewer
\end{solved}

\begin{remark}
to updated -> to update
\end{remark}
\begin{solved}
Updated as suggested by the reviewer
\end{solved}

\begin{remark}
a set of states S, the agent -> a set of states S AND the agent
\end{remark}
\begin{solved}
Updated to: ``a set of actions $A$ the agent can execute''
\end{solved}

\begin{remark} 
associated to the execution if the action -> associated to the execution OF the agent
\end{remark}
\begin{solved}
Updated as suggested by the reviewer
\end{solved}

\begin{remark}
I'd remove "Jang et al." before [9]
\end{remark}
\begin{solved}
Updated as suggested by the reviewer
\end{solved}

\begin{remark}
- Algorithm 1: where do you use R(s,a) ?
\end{remark}
\begin{solved}
The use of  the reward function takes place in the execution of the action and the update of the Q value. This was originally in the algorithm as $r$. We updated it to $R(s,a)$ as in the algorithm definition
\end{solved}

\begin{remark}
- Algorithm 1: else branch, $max_a$ is not well formatted
\end{remark}
\begin{solved}
re-formatted the $max_a$ term
\end{solved}

\begin{remark}
and agent -> an agent
\end{remark}
\begin{solved}
Rephrased sentence so the error disappeared 
\end{solved}

\begin{remark}
RL the programs -> RL programs
\end{remark}
\begin{solved}
Rephrased sentence so the error disappeared 
\end{solved}

\begin{remark}
what do you mean by "solving" three different RL programs?
\end{remark}
\begin{solved}
We rephrased the sentence to a meaningful definition of defining an agent to reach a goal in an environment
\end{solved}

\begin{remark}
the participants possible improvements -> the participants POINTED OUT possible improvements (?)
\end{remark}
\begin{solved}
Updated to: ``The participants note that, while usable, there are possible improvements in \flik's interface to improve usability.''
\end{solved}

\begin{remark}
program's execution, causes -> program's execution, WHICH causes
\end{remark}
\begin{solved}
Rephrased sentence so the error disappeared 
\end{solved}

\begin{remark}
focus in -> focus on
\end{remark}
\begin{solved}
Updated as suggested by the reviewer
\end{solved}

\begin{remark}
There are severla principles that guide the debugging process, these principles -> There are severl principles that guide the debugging process. These principles...
\end{remark}
\begin{solved}
Updated as suggested by the reviewer
\end{solved}

\begin{remark}
the things you believe to be -> the things one believes to be
\end{remark}
\begin{solved}
Updated as suggested by the reviewer
\end{solved}

\begin{remark}
your code -> the code
\end{remark}
\begin{solved}
Updated as suggested by the reviewer
\end{solved}

\begin{remark}
section 2.2, second paragraph, an itemize would improve readability.
\end{remark}

\begin{remark}
must of them -> most of them
\end{remark}
\begin{solved}
Updated as suggested by the reviewer
\end{solved}

\begin{remark}
time traveling debugger -> time travel debugger
\end{remark}
\begin{solved}
Updated as suggested by the reviewer
\end{solved}

\begin{remark}
Python debuggers, the features -> Python debuggers AND the features
\end{remark}
\begin{solved}
Updated as suggested by the reviewer
\end{solved}

\begin{remark}
What is the "Framework Fuzzing"? Explain and add some reference.
\end{remark}

\begin{remark}
a bugs life -> a bug's life
\end{remark}
\begin{solved}
Updated as suggested by the reviewer
\end{solved}

\begin{remark}
I find the phrase "without having to stop the execution" confusing. Doesn't Flik stop execution when it reaches a breakpoint or advances step by step, etc?
\end{remark}
\begin{solved}
This sentences references the full interruption of the program, while debugging the program is really just paused. We clarified the sentence by changing it to "avoiding the stop-modify-re-run loop which causes to lose the program state, or the training data already accumulated."
\end{solved}

\begin{remark}\label{rem:gridworld}
It would probably be best not to describe the GridWorld example twice. Also, Figure 2 is far from this point in the paper, which can be confusing.
\end{remark}
\begin{solved}
With the reorganization of \fref{sec:solution} we restructure the presentation of the example, leaving a minimal explanation of Gridworld and \fref{fig:gridworld} in that sections, pointing to the full presentation of thee Gridworld details in \fref{sec:evaluation}. This should make the explanations more straightforward. (\fref[vario]{par:gridworld})
\end{solved}

\begin{remark}
Page 8. Since you mention debugger commands here, it would be nice to put a reference to Table 2.
\end{remark}
\begin{solved}
Reference added as suggested
\end{solved}

\begin{remark}
The purpose of the empirical study assesing -> The purpose of the empirical study IS assesing
\end{remark}
\begin{solved}
The full sentence was rephrased to "The purpose of the user study is to assess the user experience 
and usefulness of \flik, when used to locate bugs in \ac{RL} programs"
\end{solved}

\begin{remark}
Just consider adding a bug? In practice, I guess, programs can contain several bugs that often interact with each other.
\end{remark}
\begin{solved}

\end{solved}

\begin{remark}
the the
\end{remark}
\begin{solved}
Updated as suggested by the reviewer
\end{solved}

\begin{remark}
at the beginning at the -> at the beginning OF the
\end{remark}
\begin{solved}
Updated as suggested by the reviewer
\end{solved}



%%
\subsection{Review \#2}

\begin{remark}
I suggest the authors consider literature on empirical experiments and design, such as:
- Qualitative research and evaluation methods, M. Q. Patton
- Quantitative, Qualitative and Mixed Methods Approaches, Creswell \& Creswell,
- Research Methods, Design and Analysis, Christensen, Jonson, Turner,
- Experimentation in Software Engineering, Wohlin et al.
\end{remark}

\begin{remark} 
The authors stress efficiently the difficulty and the importance of debugging RL agents, although they argue that "conventional systems typically have a linear and predictable flow [...] allowing developers to trace and debug step-by-step with ease". This is false: first even in linear and predictable systems, debugging through step-by-step can be an enormous pain, and second there are many systems that you can label as "conventional" (eg, web sites/servers, drones, iot/embedded systems, etc.) that either are not linear/predictable by nature (because, eg, of concurrency) or by context (eg, websites/iot/drones' behavior strongly depend on what happens in the uncontrolled external world).
\end{remark}

\begin{remark}\label{rem:debuggers}
There is a structure problem: subsection "2.1 what is a debugger" is mostly about what are bugs and what is debugging and then half the section concerns debuggers.
\end{remark}
\begin{solved}
In line with \fref{rem:intro} we re-structured the section "what is a debugger" into \fref{sec:deb}, a section describing the background and offering definitions for bugs and debugging (\fref[vario]{par:debuggers}) also responding to \fref{rem:debugging-process}.
\end{solved}

\begin{remark}
The debugging process is defined as "the activity that comes after testing" and seems to come from a paper. In both my experience and what I know from the literature, we debug to understand and therefore, that link with testing cannot be established like that without contextualizing.
\end{remark}
\begin{solved}
We modified the sentence given the confusion caused. We modified the definition following the references suggested by the reviewer in \fref{rem:debugging-process}
\end{solved}

\begin{remark} \label{rem:debugging-process}
The authors actually refer to the debugging process, and their statement (apparently coming from a book) refers to a specific debugging strategy known as "hypothesis testing", that is described, eg, in papers from Thomas D. Latoza (along with **other** debugging strategies such as forward reasoning and backward reasoning). Hypothesis testing is also described in books for practitioners such as "Why Programs Fail" (Zeller, 2009) or "Effective debugging: 66 specific ways to debug software and systems" (Spinellis, 2016). The "debugging process" definition, extracted from a single book, is also too restrictive: debugging "as a process" is described in broader perspectives in practitionners books such as "The science of debugging" (Telles, Hsieh, 2001), "Debugging: The 9 indispensable rules for finding even the most elusive software and hardware problems" (Agans, 2003)"Why Programs Fail" (Zeller, 2009), "Effective debugging: 66 specific ways to debug software and systems" (Spinellis, 2016).
\end{remark}
\begin{solved}
We took into consideration the suggestions made by the reviewer and following the proposed related work expanded and improved our description of the debugging process in \fref{sec:deb} (\fref[vario]{par:debugging-process})
\end{solved}

\begin{remark}\label{rem:other-langs}
"Debuggers for other programming languages", but mainly (very shortly) describes GDB and omniscient debuggers and I cannot see the "other languages" aspect.
\end{remark}
\begin{solved}
We completely re-structured the section (\fref{sec:other}) to describe relevant approaches to debugging in perspective of \flik, and present representatives of each approach in different programming languages (\fref[vario]{par:other-langs}).
\end{solved}

\begin{remark}
The authors also make general claims about back-in-time debuggers that seem off or wrong.
\end{remark}
\begin{solved}
In answering \fref{rem:other-langs}, we rewrote the description of omniscient debuggers modifying the claims about such tools.
\end{solved}

\begin{remark} 
In table 3, the authors confuse UI and GUI, and if all python debuggers allow for stepping then stepping as a criteria is not required in the table.
\end{remark}

\begin{remark}
"most of the debuggers are postmortem". Most mainstream (if not all) IDEs comes with an online debugger, so how can that "postmortem" generalization be true?
\end{remark}

\begin{remark}
The biggest problem is that the design of Flik and the rationale behind it are not presented. There is also no explicit and logical connection with the presented state of the art, or, perhaps, it mostly remain shallow and remote.
\end{remark}

\begin{remark} 
Flik should expose two features (stepping back and modifying variables) that are very common, ie, stepping back is expected from any omniscient debugger. Next, the authors presents three "main features" of Flik: storing program states and metadata, restoring program states and stepping back. These features are again standard in omniscient debuggers.
\end{remark}

\begin{remark}
Somewhere in the text pops another feature, that is live-modification of recorded execution paths, which is the only feature, as far as I know (but I don't know everything), that only exists in another debugger (namely DeloranJS). This feature should be more highlighted to distinguish Flik from others, as Flik just seems to be yet another omniscient debugger.
\end{remark}

\begin{remark}
Then some implementation details are briefly presented, not enough the grasp the originality of Flik or its potential.
\end{remark}

\begin{remark} 
My feeling is that we actually learn too few about Flik, its design and its choices/rationale, and what are the important differences with, eg, deloreanJS.
It just, as described, looks like yet another back in time debugger. Features are standard except one. The authors should stress much more what's the challenges there for RL programs, how did they approach these challenges and what are the design and implementation implications that emerged and how Flik solves these problems.
\end{remark}

\begin{remark}
The experimental design, including the kind of experiment, is not clearly stated and not sufficiently explained.
\end{remark}

\begin{remark}
There is no research question stated, and the experimental choices for investigating that missing question are therefore not explained (aside from the tasks), in particular why this design (tasks + questionnaire) is adequate for this investigation.
\end{remark}

\begin{remark} 
The "study conditions" does not specify how the task order was decided, nor if all participants undergo the same conditions (ie, the same task orders) and the possible threats to validity that are associated. For example, could the order of tasks influence the perception of participants collected in the questionnaire?
\end{remark}

\begin{remark}
The study conditions does not seem to consider any experimental variable, for example, tasks are time bounded, but in the questionnaire the authors seem to collect participants' perception about the time spent and their actual success at the tasks. Why are these information not considered as experimental variables, and measured?
\end{remark}

\begin{remark}
Why are the perceptions about time and success not contrasted with actual measures, to assess if the participants' perception are accurate, or if anything changes with the tasks? Especially, it seems that, given the time limits (ie, control of the time), that all participants use Flik, that the only actual different conditions are the tasks, and the one measurable and comparable output under that design is the success.
\end{remark}

\begin{remark} 
The survey contains 23 questions, which might be a risk considering the large number of open questions (10) at the end of the questionnaire and participants may be too tired after more than 1h of work and 13 likert-scale questions to actually provide valuable answers. This risk is never mentioned, nor how it is compensated.
\end{remark}

\begin{remark}
It is unclear about the population that the participants represent. First they are claimed to be "very experienced" but later we learn they are master and PhD students but we do not know from where they were recruited nor the means used to recruit them. In the results their experience is presented as a self-reported measure that goes from "very experienced" to "no experience", which is insufficient to characterize the population. What "very experienced" means should be characterized in terms of years and in terms of context of practice, in addition to the self-reported measure. Such precision help in understanding what population is being evaluated and thus to contextualize the results.
\end{remark}

\begin{remark}
participants mostly report a lot of experience in using Python, but not in debugging in general. This could be explained (but it is not in the paper) by the fact that RL programmers are not "standard" programmers and may come from other backgrounds, and therefore have less habits to debug. But is that on purpose? Is that because the population was drawn from RL students and therefore exposes this bias? These questions should be explained and justified in the experimental design. In addition, such information are usually collected from a demographic survey, which it seems has not been done and is not present in the design.
\end{remark}

\begin{remark} 
The results section is short, and does not go into details in analyzing the perception of participants. It reports only few information, and the open questions are not presented nor analyzed. Observations and conclusion concern only likert-scaled questions. In my opinion, open questions yield a lot of qualitative information, and, despite the weaknesses of the design, could still provide valuable insights into the usefulness of Flik.
\end{remark}

\begin{remark}
The discussion section has the same weaknesses: very short, not reflecting on the results, not discussing similar experiments (if any), not building over results (the analysis of open questions could have been interesting to discuss).
\end{remark}


\begin{remark}\label{rem:threads}
The paper does not have a threats to validity section, even though it is easy to identify numerous potential biases. Since the authors never discuss them, it is impossible to evaluate if the results are severely flawed or not.
\end{remark}
\begin{solved}
We re-worked the presentation of the user study (taking into account the previous remarks about the evaluation of our work) to mitigate any doubts that may arise about the evaluation and the results (\fref{sec:evaluation}).

Furthermore, we added a threads to validity section, \fref{sec:threads} (\fref[vario]{par:threads}), to discuss the generalization and possible bias that may arrive from our experiments in order to reduce any questions of doubts about the results.
\end{solved}

\begin{remark}
It is not mentioned if the authors obtained an ethical approval to conduct their experiment, what ethical standards they follow and if consent from participants has been collected.
\end{remark}

\begin{remark}
From the discussion, and implicitly in the conclusion, the authors conclude that Flik is/proved to be useful. I object that this conclusion cannot be drawn from the work presented in the paper, as it does not provide a solid experimental design, does not discuss threats to validity, and does not report enough empirical observations to arrive to that conclusion.
\end{remark}

\begin{remark}
the authors state as future work to optimize memory consumption. That's great, although 1) you never mentioned that aspect anywhere in the paper and it is the first time that it comes into consideration (why not in the design of Flik?) and 2) work on optimizing omniscient debuggers are numerous (eg, the work of G. Pothier), and it is not clear how you would tackle this problem under a novel perspective.
\end{remark}


\endinput


%%
% !TEX root = ../main.tex
\bibliographystyle{cas-model2-names}
%\biboptions{sort&compress}
\bibliography{local,bib/general, bib/compsci, bib/learning, bib/strings}
%\printbibliography


\end{document}


\endinput

