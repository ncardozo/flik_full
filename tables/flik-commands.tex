%\resizebox{\columnwidth}{!}{%
\begin{tabular}{c| p{14.1cm}}
\textbf{Command}               & \multicolumn{1}{c}{\textbf{Description}}  \\ 
\toprule
\spy{step_back} & Goes one step in time from the line you are\\
\spy{back_to} & Goes to the line you wish \\
\spy{variables} & Shows the local and global variables from the current execution \\
\spy{sticky} & If you lose the pretty format of \flik you can go back by typing using this command \\
\spy{args} & To see the values of the function running right now\\
\spy{setvar} & It is a way in which you can modify a variable. Note that because of optimizations of python, and variables protection this will not work in some cases \\
\midrule
\spy{p} & Evaluate and print the value of the expression given as parameter\\
\spy{pp} & Pretty-print the value of an expression\\
\spy{c} & Continue execution and only stop when a breakpoint is encountered\\
\spy{n} & Continue execution until the next line in the current function is reached or it returns \\
\spy{s} & Execute the current line and stop at the first possible occasion (either in a function that is called or in the current function) \\
\spy{unt} & Continue execution until the line with a number greater than the current one is reached. With a line number argument, continue execution until a line with a number greater or equal to that is reached \\ 
\spy{b} & With no arguments, list all breakpoints. With a line number argument, set a breakpoint at this line in the current file, or in a specific file if the \spy{filename:} prefix is used \\
\spy{w} & Print a stack trace, with the most recent frame at the bottom. An arrow indicates the current frame, which determines the context of most commands \\
\spy{u} & Move the current frame count (default one) levels up in the stack trace (to an older frame) \\
\spy{d} & Move the current frame count (default one) levels down in the stack trace (to a newer frame) \\
\spy{help} & See a list of available commands \\
\spy{q} & Quit the debugger and exit \\
\bottomrule
\end{tabular}%
%}